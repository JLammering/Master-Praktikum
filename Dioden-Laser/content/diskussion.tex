\section{Discussion}
\label{sec:Diskussion}

The goal of this experiment was to learn about the handling of a diode laser and with its help to observe the absorption spectrum of rubidium-85 and rubidium-87.
By manipulating the laser current we change the values of the internal cavity and by varying the positioning of the grating we change the behaviour of the external cavity as described in section \ref{sec:Optimizing}. Through this the right wavelength of $\lambda \approx \SI{780}{\nano\meter}$ is obtained. This fits the needed energy for observing the absortion spectrum of the two rubidium isotopes. By using the piezoelectric stack and the subtraction technique described above, a nice picture of the spectrum is achieved as shown in \ref{fig:perfect_absorption}. There we can recognise the four absorption dips 87b, 85b, 85a and 87a like in \ref{fig:dopplerandterm} only in a mirrored order. The order origins in the phase in which the piezo controller is in. It depends if its in an enlarging motion or in a shrinking motion because this changes the direction from which the wavelength are run through.
