\section{Diskussion}
\label{sec:Diskussion}

\subsection{Messung einer Stromspannungskennlinie}

Die mit der Stromspannungskennlinie bestimmte Depletionsspannung lautet
\begin{align}
  U_\text{Dep,I(U)} = \SI{65.09}{\volt}
  \label{eqn:UdepI}
\end{align}
und liegt im Bereich der vom Hersteller genannten Depletionsspannung
\begin{align}
  U_\text{Dep,Hersteller} \approx \SI{60}{\volt} - \SI{80}{\volt}.
\end{align}
Jedoch ist an Abbildung \ref{fig:stromspannungskennlinie} erkennbar, dass
der Leckstrom erst bei deutlich höheren Vorspannung $> \SI{100}{\volt}$ sichtbar
linear mit der Spannung ansteigt, weshalb es sinnvoll ist, die Vorspannung in den
anschließenden Messreihen höher als \eqref{eqn:UdepI} einzustellen.

\subsection{Pedestal, Common Mode Shift und Noise}

Der bestimmte mittlere Pedestal liegt im Durchschnitt bei
\begin{align}
  \text{Pedestal}_\text{average} = 509.122 \pm 2.210 \, \text{ADC}
\end{align}
und damit am erwarteten Wert $\SI{500}{\text{ADC}}$.

Der Common Mode Shift ist in Abbildung \ref{fig:cms} als normiertes Histogramm dargestellt.
Die gaußförmige Verteilung durch den zentralen Grenzwertsatz ist zwar erkennbar, allerdings
mit sichtbaren statistischen Schwankungen. Das ist darauf zurückzuführen, dass lediglich
1000 Events für diese Messung aufgenommen wurden.

Der sich ergebende mittlere Noise liegt bei
\begin{align}
  \text{Noise}_\text{average} = 2.119 \pm 0.081 \, \text{ADC}.
  \label{eqn:noise}
\end{align}
und schwankt mit einem relativen Fehler von $\SI{3.8}{\percent}$ schwach für
jeden Streifen.

\subsection{Kalibrationsmessungen}

In Abbildung \ref{fig:kalibration} ist erkennbar, dass die Kalibrationskurven der Kanäle 20,40,60,80,100
für eine Vorspannung oberhalb der Depletionsspannung quasi alle aufeinander liegen. Das Polynom vierten Grades
nähert diese im Bereich $\text{charge} = [0,250000 \symup{e}]$ ausgezeichnet an, was ebenfalls in der
Abbildung ersichtlich ist.

Die Kalibrationskurve bei $\SI{0}{\volt}$ ist in Abbildung \ref{fig:kalibration2} mit den anderen Kurven
in einem großen Bereich (in dem die Diskrepanz der Messwerte am ehesten erkennbar ist) dargestellt.
An dieser Graphik ist erkennbar, dass das Einstellen der Depletionsspannung
eine wichtige Rolle spielt, da Signale bei der Umrechnung ansonsten um etwa $\SI{5}{\text{ADC}}$ abweichen, was
beispielsweise doppelt so viel wie der berechneten durchschnittliche Noise \eqref{eqn:noise} ist.

\subsection{Vermessung der Streifensensoren mittels eines Lasers}

Aus Abbildung \ref{fig:pitch} ist entnehmbar, dass der Laser über die Kanäle 91 und 92 bewegt wurde.
Die berechnete pitch liegt bei
\begin{align}
  \text{pitch}_{92-91} = \SI{161.69}{\micro\meter}
\end{align}
und weicht damit relativ gering um $\SI{1}{\percent}$ von der vom Hersteller genannten pitch
\begin{align}
  \text{pitch}_\text{Hersteller} = \SI{160}{\micro\meter}
\end{align}
ab. Der Sitz der Kathode ist zur Vermessung ionisierender Strahlung irrelevant, da diese Strahlung
kaum von der Metallisierung reflektiert wird. Beim Laserstrahl wird genau die Tatsache, dass das
Licht reflektiert wird, zur Vermessung des Streifensensors ausgenutzt und zwar erfolgreich,
wie an der geringen Abweichung erkennbar ist.

\subsection{Charge Collection Efficiency Messungen mit dem Laser}

Die mit der Charge Collection Efficiency Messung mit dem Laser berechnete Depletionsspannung
\begin{align}
  U_\text{dep,ccel} = \SI{110(5)}{\volt}
\end{align}
ist wesentlich größer als die mit der Stromspannungskennlinie bestimmte Depletionsspannung
\begin{align}
  U_\text{Dep,I(U)} = \SI{65.09}{\volt}.
\end{align}
Das ist darauf zurückzuführen, dass die Leckstrommessung zwar ein guter Indikator für das Erreichen der
Depletionsspannung ist, aber die Vorspannung, die tatsächlich für die Signalmessung eingestellt werden sollte,
größer ist. Der Grund dafür liegt darin, dass beim Erreichen der Depletionsspannung das äußere elektrische Feld noch nicht
ausreichend stark ist, um jedes entstehende Elektron-Loch-Paar an der Rekombination zu hindern. Der Leckstrom steigt aus diesem
Grund auch weiter an, wenn die Spannung über die Depletionsspannung hinausgeht: ein immer größerer Anteil der durch
thermische Anregung erzeugten Elektron-Loch-Paaren erreicht die Pole und rekombiniert nicht.

\subsection{Charge Collection Efficiency Messungen mit der Quelle}

Die mit der Charge Collection Efficiency Messungen mit der Quelle berechnete Depletionsspannung
\begin{align}
  U_\text{depl,cceq} = \SI{120(5)}{\volt}
\end{align}
weicht mit $\SI{9}{\percent}$ leicht von der Messung mit dem Laser ab. Was außerdem in Abbildung
\ref{fig:cceq} heraussticht, ist, dass das Plateau eine leichte Steigung besitzt. Das ist darauf zurückzuführen, dass
das Elektron aus dem radioaktiven $\beta$-Zerfall im Gegensatz zu den Photonen aus dem Laser durch das
elektrische Feld im Detektor leicht abgebremst wird. Wird die Spannung erhöht, so steigt die elektrische Feldstärke
und das Elektron hält sich länger im Detektor auf, sodass sich die Anzahl an Stoßprozessen erhöht und die Energiedeposition steigt.

\subsection{Großer Quellenscan}
