\section{Auswertung}
\label{sec:Auswertung}

\subsection{Messung einer Stromspannungskennlinie}

Die Messwerte der Stromspannungskennlinie sind in Tabelle \ref{tab:stromspannungskennlinie} aufgelistet
und in Abbildung \ref{fig:stromspannungskennlinie} graphisch dargestellt. An der Stelle der Depletionsspannung
wird in der Kennlinie ein Knick erwartet, da der Leckstrom bei voll depletiertem Sensor schwächer mit der Spannung
ansteigt. Ein Knick entspricht mathematisch ausgedrückt einem Maximum in der zweiten Ableitung einer Funktion.
Die Messwerte werden daher zunächst polynomiell angenähert und im Anschluss daran wird die Nullstelle der dritte Ableitung des
Ausgleichspolynoms bestimmt. Für das Polynom der Form
\begin{align}
  I(U) = a U^4 + b U^3 + c U^2 + d U + e
\end{align}
ergeben sich die Parameter
\begin{align*}
  a &= \SI{3.46e-10}{\micro\ampere\per\volt\tothe{4}} &\quad b &= \SI{-9.00e-08}{\micro\ampere\per\volt\tothe{3}} &\quad c&= \SI{-1.10e-05}{\micro\ampere\per\volt\tothe{2}} \\
  d &= \SI{5.2e-03}{\micro\ampere\per\volt} &\quad e &= \SI{0.77}{\micro\ampere} &\quad \phantom{f}&\phantom{=10}.
\end{align*}
Der Knick wird an der Stelle
\begin{align}
  U_\text{Knick} = U_\text{Dep,I(U)} = -\frac{b}{4a} = \SI{65.09}{\volt}
\end{align}
lokalisiert.

\begin{figure}
  \centering
  \includegraphics{build/stromspannungskennlinie.pdf}
  \caption{Messwerte und Ausgleichspolynom der Stromspannungskennlinie.}
  \label{fig:stromspannungskennlinie}
\end{figure}

\subsection{Pedestal, Common Mode Shift und Noise}

In Abbildung \ref{fig:pedestals} sind die Pedestals für alle 128 Streifen graphisch dargestellt.
Der berechnete Mittelwert lautet
\begin{align}
   \text{Pedestal}_\text{average} = 509.122 \pm 2.210 \, \text{ADC}.
\end{align}

\begin{figure}
  \centering
  \includegraphics{build/Pedestal.pdf}
  \caption{Berechneter Pedestal in ADC für alle Streifen $i$.}
  \label{fig:pedestals}
\end{figure}

Der Common Mode Shift wird für alle 1000 Events aus dem Pedestal berechnet.
Die Werte sind in Abbildung \ref{fig:cms} als normiertes Histogramm dargestellt.
Aufgrund des zentralen Grenzwertsatzes hat das Histogramm die Form einer Normalverteilung um $\text{CMS} = 0$.
Ein Funktionenfit mit
\begin{align}
  p_k(\text{CMS}) = \frac{1}{\sqrt{2 \pi \sigma^2}} \text{exp} \left(-\frac{(\text{CMS}/\text{ADC})^2}{2\sigma^2}\right)
  \label{eqn:gauß}
\end{align}
liefert die Standardabweichung
\begin{align}
  \sigma = 2.681.
\end{align}

\begin{figure}
  \centering
  \includegraphics{build/Common_Mode_Shift.pdf}
  \caption{Berechneter Common Mode Shift in ADC dargestellt als Wahrscheinlichkeitsverteilung $p_k$.}
  \label{fig:cms}
\end{figure}

In Abbildung \ref{fig:noise} ist der aus dem Pedestal und dem Common Mode Shift berechnete Noise für jeden Kanal
dargestellt. Es ergibt sich ein Mittelwert von
\begin{align}
  \text{Noise}_\text{average} = 2.119 \pm 0.081 \, \text{ADC}.
\end{align}

\begin{figure}
  \centering
  \includegraphics{build/Noise.pdf}
  \caption{Berechneter Noise in ADC für alle Streifen $i$.}
  \label{fig:noise}
\end{figure}

\subsection{Kalibrationsmessungen}

Mittels einer Delaymessung im Calibration Run wird eine optimale Verzögerung von
\begin{align}
  1
\end{align}
bestimmt und im Calibration Fenster eingetragen. Für eine Vorspannung oberhalb der
Depletionsspannung werden fünf Kalibrationskurven für die Kanäle 20, 40, 60, 80 und 100 aufgenommen. In Abbildung \ref{fig:kalibration}
ist der Mittelwert dieser Kurven aufgetragen. Außerdem sind die nicht gemittelten
Messwerte im Graphen hinzugefügt und es ist erkennbar, dass diese kaum vom Mittelwert abweichen und
alle aufeinander liegen. Desweiteren wird die gemittelte Kalibrationskurve polynomiell angenähert.
Das resultierende Polynom
\begin{align}
  K(\text{charge}) = k_1 \text{charge}^4 + k_2 \text{charge}^3 + k_3 \text{charge}^2 + k_4 \text{charge} + k_5
  \label{eqn:kalibrationspolynom}
\end{align}
ist ebenfalls in Abbildung \ref{fig:kalibration} dargestellt und es ergeben sich die Parameter
\begin{align*}
  k_1 &= \SI{-1.56e-20}{\text{ADC}e^{-4}} &\quad k_2 &= \SI{3.91e-14}{\text{ADC}e^{-3}} \\
  k_3 &= \SI{-2.04e-08}{\text{ADC}e^{-2}} &\quad k_4 &= \SI{3.99e-03}{\text{ADC}e^{-1}} \\
  k_5 &= \SI{-3.71}{\text{ADC}}. &\quad \phantom{f}&\phantom{=10}
\end{align*}
Es wird ebenfalls eine Kalibrationskurve für eine Vorspannung von $\SI{0}{\volt}$ aufgenommen. In Abbildung
\ref{fig:kalibration2} ist erkennbar, dass diese in einem großen Bereich von den anderen Kalibrationskurven,
bzw. deren Mittelwert, abweicht.

\begin{figure}
  \centering
  \includegraphics{build/kalibration.pdf}
  \caption{Plot.}
  \label{fig:kalibration}
\end{figure}

\begin{figure}
  \centering
  \includegraphics{build/kalibration2.pdf}
  \caption{Plot.}
  \label{fig:kalibration2}
\end{figure}

\subsection{Vermessung der Streifensensoren mittels eines Lasers}

Mit der Option Laser Sync. wird die optimale Verzögerung zwischen Lasersignal und
Chipauslese bestimmt. Es ergibt sich
\begin{align}
  1.
\end{align}
In Abbildung \ref{fig:pitch} sind die über die Laserverschiebung gemittelten Signal-Messwerte für jeden Streifen
dargestellt. Es ist erkennbar, dass der Laser über die Streifen 91 und 92 bewegt wurde,
da sich der Signal-Messwert dort im Mittel deutlich von den anderen Werten abhebt.
Die ungemittelten Messwerte für diese beiden Streifen sind in Abbildung \ref{fig:pitch2} dargestellt.
Die Messwerte zu beiden Streifen beschreiben eine Kurve mit zwei herausragenden Maxima.
Das Minimum zwischen diesen beiden Maxima beschreibt jeweils den Fall, dass der Laser genau über dem jeweiligen
Streifen positioniert ist. Um die pitch zwischen den Streifen zu bestimmen, wird zunächst das Minimum der Kanäle 91 und 92
jeweils durch eine Parabel angenähert. Es ergeben sich für die Funktion
\begin{align}
  \text{Signal}_{\text{min,}i}(x) = u_i x^2 + v_i x + w_i
\end{align}
die Parameter
\begin{align*}
  u_{91} &= \SI{0.1009}{\text{ADC}\meter\tothe{-2}} &\quad v_{91} &= \SI{-19.35}{\text{ADC}\meter\tothe{-1}} &\quad w_{91} &= \SI{927.5}{\text{ADC}} \\
  u_{92} &= \SI{0.0894}{\text{ADC}\meter\tothe{-2}} &\quad v_{92} &= \SI{-46.08}{\text{ADC}\meter\tothe{-1}} &\quad w_{92} &= \SI{5934.6}{\text{ADC}}.
\end{align*}
Die pitch zwischen den Streifen (hier 91 und 92) ergibt sich aus der Differenz der Scheitelpunkte der Parabeln gemäß
\begin{align}
  \text{pitch}_{92-91} = \frac{-v_{92}}{2u_{92}} - \frac{-v_{91}}{2u_{91}} = \SI{161.69}{\micro\meter}.
\end{align}
Um außerdem die Ausdehnung des Lasers zu bestimmen wird der Peak aus Abbildung \ref{fig:pitch} genauer betrachtet.
In Abbildung \ref{fig:Ausdehnung} sind die über die Laserverschiebung gemittelten Signal-Messwerte für die Streifen
88 bis 95 dargestellt. Es wird ein Funktionenfit mit einer Normalverteilung
\begin{align}
  P_\text{Laser}(z) = \frac{1}{\sqrt{2 \pi \sigma_\text{ausd}^2}} \text{exp}\left(-\frac{(z-\mu_\text{ausd})^2}{2\sigma_\text{ausd}^2}\right)
\end{align}
durchgeführt. Dabei ergibt sich der Mittelwert
\begin{align}
  \mu_\text{ausd} = 91.46
\end{align}
und die Standardabweichung
\begin{align}
  \sigma_\text{ausd} = 0.8303.
\end{align}
Es wird angenommen, dass die Ausdehnung des Lasers in etwa der $2\sigma$-Umgebung der Gaußfunktion abzüglich der Distanz, die der Laser bewegt wurde, entspricht.
Insgesamt folgt daher
\begin{align}
  \text{Ausdehnung} = 4\sigma_\text{ausd} \cdot \text{pitch} - \SI{35}{\micro\meter} = \SI{501.96}{\micro\meter}.
\end{align}

\begin{figure}
  \centering
  \includegraphics{build/pitch.pdf}
  \caption{Plot.}
  \label{fig:pitch}
\end{figure}

\begin{figure}
  \centering
  \includegraphics{build/pitch2.pdf}
  \caption{Plot.}
  \label{fig:pitch2}
\end{figure}

\begin{figure}
  \centering
  \includegraphics{build/Ausdehnung.pdf}
  \caption{Plot.}
  \label{fig:Ausdehnung}
\end{figure}

\subsection{Charge Collection Efficiency Messungen mit dem Laser}

Um einen Zusammenhang zwischen der Charge Collection Efficiency und der Vorspannung zu erhalten,
wird eines der Maxima um Kanal 91 bei verschiedenen Spannungen ausgemessen. Die maximal mögliche Effizienz,
die erreicht werden kann stellt sich ab der Depletionsspannung ein. Um diese Spannung zu bestimmen muss also
die Stelle, ab der die Signal-Messwerte ein Plateau darstellen bestimmt werden. Die über alle Events gemittelten
Signal-Messwerte sind in Abbildung \ref{fig:ccel} gegen die Spannung aufgetragen. Es wird eine lineare Ausgleichsrechnung
mit den Messwerten, die oberhalb der Spannungen $\SI{90}{\volt}$, $\SI{100}{\volt}$ bzw. $\SI{110}{\volt}$ liegen, da dort gemäß Abbildung \ref{fig:ccel}
der Übergang zum Plateau erwartet wird.
Für die Geradengleichung
\begin{align}
  \text{CCEL-Signal}_i(U) = m_i U + b_i
\end{align}
ergeben sich die Koeffizienten
\begin{align*}
  m_{\geq90}  &= \SI{0.0083(30)}{\text{ADC}\volt\tothe{-1}} &\quad b_{\geq90}  &= \SI{133.6(5)}{\text{ADC}} \\
  m_{\geq100} &= \SI{0.0042(24)}{\text{ADC}\volt\tothe{-1}} &\quad b_{\geq100} &= \SI{134.3(4)}{\text{ADC}} \\
  m_{\geq110} &= \SI{0.0012(19)}{\text{ADC}\volt\tothe{-1}} &\quad b_{\geq110} &= \SI{134.8(3)}{\text{ADC}}.
\end{align*}
Der Plateaubeginn wird auf die Spannung mit der geringsten Steigung und geringstem Fehler in der Steigung zurückgeführt.
Daher ergibt sich in dieser Messung eine Depletionsspannung von etwa
\begin{align}
  U_\text{dep,ccel} = \SI{110(5)}{\volt}.
\end{align}
Zur Bestimmung der Eindringtiefe des Lasers werden zunächst alle Messwerte auf den Plateaumittelwert normiert
\begin{align}
  \text{Plateau}_\text{average} = \SI{135.02}{\text{ADC}},
\end{align}
der sich aus dem Mittelwert aller Messwerte oberhalb der Depletionsspannung ergibt.
In Abbildung \ref{fig:eindringtiefe} ist die resultierende Charge Collection Efficiency für alle aufgenommenen Spannungen unterhalb bzw.
auf der Depletionsspannung dargestellt. Es wird ein Funktionenfit gemäß Gleichung \eqref{eqn:cce} durchgeführt, woraus sich
die Eindringtiefe
\begin{align}
  a_\text{Laser} = \SI{106.43}{\micro\meter}
\end{align}
ergibt.

\begin{figure}
  \centering
  \includegraphics{build/ccel.pdf}
  \caption{Plot.}
  \label{fig:ccel}
\end{figure}

\begin{figure}
  \centering
  \includegraphics{build/eindringtiefe.pdf}
  \caption{Plot.}
  \label{fig:eindringtiefe}
\end{figure}

\subsection{Charge Collection Efficiency Messungen mit der Quelle}

Die gegebenen Messwerte werden im ersten Schritt von ADC in $\si{\kilo\electronvolt}$ umgerechnet.
Da zu dem Kalibrationspolynom \eqref{eqn:kalibrationspolynom} keine Umkehrfunktion gebildet werden kann,
wird die Umrechnung durch das Erstellen von Bins durchgeführt. Insgesamt wird die charge im relevanten Bereich
$[0,260000\mathrm{e}]$ in 10000 Bins unterteilt und zu jedem dieser Bins der entsprechende Wert in ADC mittels
des Kalibrationspolynoms \eqref{eqn:kalibrationspolynom} bestimmt. Die Umrechnung von ADC in $\si{\kilo\electronvolt}$
erfolgt dann durch Finden des jeweiligen Bins in ADC, Zuordnen des jeweiligen Bins in charge und als letztes Multiplizieren
mit dem Faktor $\num{3.6e-3}$, da zur Erzeugung eines Elektron-Loch-Paares im Detektor $\SI{3.6}{\electronvolt}$
benötigt werden. Anschließend werden alle Energie-Werte eines Events summiert, woraus sich die jeweilige Clusterenergie ergibt.
Im letzten Schritt werden die Clusterenergien für jede Spannung jeweils über alle Events gemittelt. Die resultierenden
gemittelten Clusterenergien sind in Abbildung \ref{fig:cceq} in Abhängigkeit von der Spannung aufgetragen.
Der Knick wird hier gemäß Abbildung \ref{fig:cceq} bei den Vorspannungen $U = \SI{120}{\volt}$, $U = \SI{130}{\volt}$ und
$U = \SI{140}{\volt}$ vermutet. Es wird erneut eine lineare Ausgleichsrechnung durchgeführt.
Für die Geradengleichung
\begin{align}
  \text{CCEQ-Energie}_j(U) = g_j U + h_j
\end{align}
ergeben sich die Parameter
\begin{align*}
  g_{\geq140} &= \SI{0.018(12)}{\text{ADC}\volt\tothe{-1}} &\quad b_{\geq140} &= \SI{102.1(20)}{\text{ADC}} \\
  g_{\geq130} &=  \SI{0.024(9)}{\text{ADC}\volt\tothe{-1}} &\quad b_{\geq130} &= \SI{101.1(16)}{\text{ADC}} \\
  g_{\geq120} &=  \SI{0.024(7)}{\text{ADC}\volt\tothe{-1}} &\quad b_{\geq120} &= \SI{101.1(12)}{\text{ADC}}.
\end{align*}
Die geringste Steigung ist in diesem Fall dem Plateau 140 zugeordnet, jedoch ist der Fehler hier vergleichsweise groß.
Bei Betrachtung der Messwerte wird daher interpretiert, dass das Plateau etwa bei
\begin{align}
  U_\text{depl,cceq} = \SI{120}{\volt}
\end{align}
beginnt und nicht flach ist, sondern eine leichte Steigung von etwa
\begin{align}
  g_{\geq120} =  \SI{0.024(7)}{\text{ADC}\volt\tothe{-1}}
\end{align}
besitzt.

\begin{figure}
  \centering
  \includegraphics{build/cceq.pdf}
  \caption{Plot.}
  \label{fig:cceq}
\end{figure}

\subsection{Großer Quellenscan}



%\begin{figure}
%  \centering
%  \includegraphics{plotphase.pdf}
%  \caption{Plot.}
%  \label{fig:plot}
%\end{figure}

%Tabelle für copy and paste:
%\begin{table}[h]
%  \centering
%  \begin{tabular}{S S}
%    \toprule
%    {$k$} & {$U\:/\:\si{\milli\volt}$}\\
%    \midrule
%    1 & 637.2\\
%    3 & 212.4\\
%    5 & 127.4\\
%    7 & 91.03\\
%    9 & 70.8\\
%    \bottomrule
%  \end{tabular}
%  \caption{Amplituden Rechteckspannung.}
%  \label{tab:rechtampl}
%\end{table}


\begin{table}[h]
  \centering
  \begin{tabular}{S S}
    \toprule
    {$U\:/\:\si{\volt}$} & {$I\:/\:\si{\micro\ampere}$} \\
    \midrule
    199 & 1.20 \\
    190 & 1.20 \\
    180 & 1.19 \\
    170 & 1.19 \\
    160 & 1.18 \\
    150 & 1.18 \\
    140 & 1.17 \\
    130 & 1.16 \\
    120 & 1.15 \\
    110 & 1.14 \\
    100 & 1.12 \\
    90  & 1.11 \\
    80  & 1.09 \\
    70  & 1.06 \\
    60  & 1.04 \\
    50  & 1.00 \\
    40  & 0.96 \\
    30  & 0.91 \\
    20  & 0.86 \\
    10  & 0.82 \\
    0   & 0.78 \\
    \bottomrule
  \end{tabular}
  \caption{Messwerte der Stromspannungskennlinie zur Bestimmung der Depletionsspannung.}
  \label{tab:stromspannungskennlinie}
\end{table}
