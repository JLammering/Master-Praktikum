\section{Auswertung}
\label{sec:Auswertung}

\subsection{Messung einer Stromspannungskennlinie}

Die Messwerte der Stromspannungskennlinie sind in Tabelle \ref{tab:stromspannungskennlinie} aufgelistet
und in Abbildung \ref{fig:stromspannungskennlinie} graphisch dargestellt. An der Stelle der Depletionsspannung
wird in der Kennlinie ein Knick erwartet, da der Leckstrom bei voll depletiertem Sensor schwächer mit der Spannung
ansteigt. Ein Knick entspricht mathematisch ausgedrückt einem Maximum in der zweiten Ableitung einer Funktion.
Die Messwerte werden daher zunächst polynomiell angenähert und im Anschluss daran wird die Nullstelle der dritte Ableitung des
Ausgleichspolynoms bestimmt. Für das Polynom der Form
\begin{align}
  I(U) = a U^4 + b U^3 + c U^2 + d U + e
\end{align}
ergeben sich die Parameter
\begin{align*}
  a &= \SI{3.46e-10}{\micro\ampere\per\volt\tothe{4}} &\quad b &= \SI{-9.00e-08}{\micro\ampere\per\volt\tothe{3}} &\quad c&= \SI{-1.10e-05}{\micro\ampere\per\volt\tothe{2}} \\
  d &= \SI{5.2e-03}{\micro\ampere\per\volt} &\quad e &= \SI{0.77}{\micro\ampere} &\quad \phantom{f}&\phantom{=10}.
\end{align*}
Der Knick wird an der Stelle
\begin{align}
  U_\text{Knick} = U_\text{Dep,I(U)} = -\frac{b}{4a} = \SI{65.09}{\volt}
\end{align}
lokalisiert.

\begin{figure}
  \centering
  \includegraphics{build/stromspannungskennlinie.pdf}
  \caption{Messwerte und Ausgleichspolynom der Stromspannungskennlinie.}
  \label{fig:stromspannungskennlinie}
\end{figure}

\subsection{Pedestal, Common Mode Shift und Noise}

In Abbildung \ref{fig:pedestals} sind die Pedestals für alle 128 Streifen graphisch dargestellt.
Der berechnete Mittelwert lautet
\begin{align}
   \text{Pedestal}_\text{average} = 509.122 \pm 2.210 \, \text{ADC}.
\end{align}

\begin{figure}
  \centering
  \includegraphics{build/Pedestal.pdf}
  \caption{Berechneter Pedestal in ADC für alle Streifen $i$.}
  \label{fig:pedestals}
\end{figure}

Der Common Mode Shift wird für alle 1000 Events aus dem Pedestal berechnet.
Die Werte sind in Abbildung \ref{fig:cms} als normiertes Histogramm dargestellt.
Aufgrund des zentralen Grenzwertsatzes hat das Histogramm die Form einer Normalverteilung um $\text{CMS} = 0$.
Ein Funktionenfit mit
\begin{align}
  p_k(\text{CMS}) = \frac{1}{\sqrt{2 \pi \sigma^2}} \symup{e}^{-\frac{\text{CMS}/\text{ADC}}{2\sigma^2}}
\end{align}
liefert die Standardabweichung
\begin{align}
  \sigma = 2.681.
\end{align}

\begin{figure}
  \centering
  \includegraphics{build/Common_Mode_Shift.pdf}
  \caption{Berechneter Common Mode Shift in ADC dargestellt als Wahrscheinlichkeitsverteilung $p_k$.}
  \label{fig:cms}
\end{figure}

In Abbildung \ref{fig:noise} ist der aus dem Pedestal und dem Common Mode Shift berechnete Noise für jeden Kanal
dargestellt. Es ergibt sich ein Mittelwert von
\begin{align}
  \text{Noise}_\text{average} = 2.119 \pm 0.081 \, \text{ADC}.
\end{align}

\begin{figure}
  \centering
  \includegraphics{build/Noise.pdf}
  \caption{Berechneter Noise in ADC für alle Streifen $i$.}
  \label{fig:noise}
\end{figure}

\subsection{Kalibrationsmessungen}

Mittels einer Delaymessung im Calibration Run wird eine optimale Verzögerung von
\begin{align}
  1
\end{align}
bestimmt und im Calibration Fenster eingetragen. Für eine Vorspannung oberhalb der
Depletionsspannung werden fünf Kalibrationskurven für die Kanäle 20, 40, 60, 80 und 100 aufgenommen. In Abbildung \ref{fig:kalibration}
ist der Mittelwert dieser Kurven aufgetragen. Außerdem sind die nicht gemittelten
Messwerte im Graphen hinzugefügt und es ist erkennbar, dass diese kaum vom Mittelwert abweichen und
alle aufeinander liegen. Desweiteren wird die gemittelte Kalibrationskurve polynomiell angenähert.
Das resultierende Polynom
\begin{align}
  K(\text{charge}) = k_1 \text{charge}^4 + k_2 \text{charge}^3 + k_3 \text{charge}^2 + k_4 \text{charge} + k_5
\end{align}
ist ebenfalls in Abbildung \ref{fig:kalibration} dargestellt und es ergeben sich die Parameter
\begin{align*}
  k_1 &= \SI{-1.56e-20}{\text{ADC}e^{-4}} &\quad k_2 &= \SI{3.91e-14}{\text{ADC}e^{-3}} \\
  k_3 &= \SI{-2.04e-08}{\text{ADC}e^{-2}} &\quad k_4 &= \SI{3.99e-03}{\text{ADC}e^{-1}} \\
  k_5 &= \SI{-3.71}{\text{ADC}} &\quad \phantom{f}&\phantom{=10}.
\end{align*}
Es wird ebenfalls eine Kalibrationskurve für eine Vorspannung von $\SI{0}{\volt}$ aufgenommen. In Abbildung
\ref{fig:kalibration2} ist erkennbar, dass diese in einem großen Bereich von den anderen Kalibrationskurven,
bzw. deren Mittelwert, abweicht.

\begin{figure}
  \centering
  \includegraphics{build/kalibration.pdf}
  \caption{Plot.}
  \label{fig:kalibration}
\end{figure}

\begin{figure}
  \centering
  \includegraphics{build/kalibration2.pdf}
  \caption{Plot.}
  \label{fig:kalibration2}
\end{figure}

\begin{figure}
  \centering
  \includegraphics{plotphase.pdf}
  \caption{Plot.}
  \label{fig:plot}
\end{figure}

Tabelle für copy and paste:
\begin{table}[h]
  \centering
  \begin{tabular}{S S}
    \toprule
    {$k$} & {$U\:/\:\si{\milli\volt}$}\\
    \midrule
    1 & 637.2\\
    3 & 212.4\\
    5 & 127.4\\
    7 & 91.03\\
    9 & 70.8\\
    \bottomrule
  \end{tabular}
  \caption{Amplituden Rechteckspannung.}
  \label{tab:rechtampl}
\end{table}


\begin{table}[h]
  \centering
  \begin{tabular}{S S}
    \toprule
    {$U\:/\:\si{\volt}$} & {$I\:/\:\si{\micro\ampere}$} \\
    \midrule
    199 & 1.20 \\
    190 & 1.20 \\
    180 & 1.19 \\
    170 & 1.19 \\
    160 & 1.18 \\
    150 & 1.18 \\
    140 & 1.17 \\
    130 & 1.16 \\
    120 & 1.15 \\
    110 & 1.14 \\
    100 & 1.12 \\
    90  & 1.11 \\
    80  & 1.09 \\
    70  & 1.06 \\
    60  & 1.04 \\
    50  & 1.00 \\
    40  & 0.96 \\
    30  & 0.91 \\
    20  & 0.86 \\
    10  & 0.82 \\
    0   & 0.78 \\
    \bottomrule
  \end{tabular}
  \caption{Messwerte der Stromspannungskennlinie zur Bestimmung der Depletionsspannung.}
  \label{tab:stromspannungskennlinie}
\end{table}
