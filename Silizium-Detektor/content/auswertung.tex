\section{Auswertung}
\label{sec:Auswertung}

\subsection{Messung einer Stromspannungskennlinie}

Die Messwerte der Stromspannungskennlinie sind in Tabelle \ref{tab:stromspannungskennlinie} aufgelistet
und in Abbildung \ref{fig:stromspannungskennlinie} graphisch dargestellt. An der Stelle der Depletionsspannung
wird in der Kennlinie ein Knick erwartet, da der Leckstrom bei voll depletiertem Sensor schwächer mit der Spannung
ansteigt. Ein Knick entspricht mathematisch ausgedrückt einem Maximum in der zweiten Ableitung einer Funktion.
Die Messwerte werden daher zunächst polynomiell angenähert und im Anschluss daran wird die Nullstelle der dritte Ableitung des
Ausgleichspolynoms bestimmt. Für das Polynom der Form
\begin{align}
  I(U) = a U^4 + b U^3 + c U^2 + d U + e
\end{align}
ergeben sich die Parameter
\begin{align*}
  a &= \SI{35(15)e-11}{\micro\ampere\per\volt\tothe{4}} &\quad b &= \SI{-9(6)e-08}{\micro\ampere\per\volt\tothe{3}} \\
  c&= \SI{-11(8)e-06}{\micro\ampere\per\volt\tothe{2}} &\quad d &= \SI{5,2(4)e-03}{\micro\ampere\per\volt} \\
  e &= \SI{0,772(5)}{\micro\ampere} &\quad \phantom{f}&\phantom{=10}.
\end{align*}
Der Knick wird an der Stelle
\begin{align}
  U_\text{Knick} = U_\text{Dep,I(U)} = -\frac{b}{4a} = \SI{7(5)e01}{\volt}
\end{align}
lokalisiert.

\begin{figure}
  \centering
  \includegraphics{build/stromspannungskennlinie.pdf}
  \caption{Messwerte und Ausgleichspolynom der Stromspannungskennlinie.}
  \label{fig:stromspannungskennlinie}
\end{figure}

\subsection{Pedestal, Common Mode Shift und Noise}

In Abbildung \ref{fig:pedestals} sind die Pedestals für alle 128 Streifen graphisch dargestellt.
Der berechnete Mittelwert lautet
\begin{align}
   \text{Pedestal}_\text{average} = 509,122 \pm 2,210 \, \text{ADC}.
\end{align}

\begin{figure}
  \centering
  \includegraphics{build/Pedestal.pdf}
  \caption{Berechneter Pedestal in ADC für alle Streifen $i$.}
  \label{fig:pedestals}
\end{figure}

Der Common Mode Shift wird für alle 1000 Events aus dem Pedestal berechnet.
Die Werte sind in Abbildung \ref{fig:cms} als normiertes Histogramm dargestellt.
Aufgrund des zentralen Grenzwertsatzes hat das Histogramm die Form einer Normalverteilung um $\text{CMS} = 0$.
Ein Funktionenfit mit
\begin{align}
  p_k(\text{CMS}) = \frac{1}{\sqrt{2 \pi \sigma^2}} \text{exp} \left(-\frac{(\text{CMS}/\text{ADC})^2}{2\sigma^2}\right)
  \label{eqn:gauß}
\end{align}
liefert die Standardabweichung
\begin{align}
  \sigma = \num{2,68(7)}
\end{align}

\begin{figure}
  \centering
  \includegraphics{build/Common_Mode_Shift.pdf}
  \caption{Berechneter Common Mode Shift in ADC dargestellt als Wahrscheinlichkeitsverteilung $p_k$.}
  \label{fig:cms}
\end{figure}

In Abbildung \ref{fig:noise} ist der aus dem Pedestal und dem Common Mode Shift berechnete Noise für jeden Kanal
dargestellt. Es ergibt sich ein Mittelwert von
\begin{align}
  \text{Noise}_\text{average} = \SI{2.119(81)}{ADC}.
\end{align}

\begin{figure}
  \centering
  \includegraphics{build/Noise.pdf}
  \caption{Berechneter Noise in ADC für alle Streifen $i$.}
  \label{fig:noise}
\end{figure}

\subsection{Kalibrationsmessungen}
\label{sec:kalibration}

Mittels einer Delaymessung im Calibration Run wird eine optimale Verzögerung von
\begin{align}
  \text{Kalibration Delay} = \SI{64.1}{\nano\second}
\end{align}
bestimmt und im Calibration Fenster eingetragen. Für eine Vorspannung oberhalb der
Depletionsspannung werden fünf Kalibrationskurven für die Kanäle 20, 40, 60, 80 und 100 aufgenommen. In Abbildung \ref{fig:kalibration}
ist der Mittelwert dieser Kurven aufgetragen. Außerdem sind die nicht gemittelten
Messwerte im Graphen hinzugefügt und es ist erkennbar, dass diese kaum vom Mittelwert abweichen und
alle aufeinander liegen. Daher wird angenommen, dass die Kalibrationskurve für alle Kanäle gleich verläuft. Desweiteren wird die gemittelte Kalibrationskurve polynomiell angenähert.
Das resultierende Polynom
\begin{align}
  K(\text{charge}) = k_1 \text{charge}^4 + k_2 \text{charge}^3 + k_3 \text{charge}^2 + k_4 \text{charge} + k_5
  \label{eqn:kalibrationspolynom}
\end{align}
ist ebenfalls in Abbildung \ref{fig:kalibration} dargestellt und es ergeben sich die Parameter
\begin{align*}
  k_1 &= \SI{-1.56(34)e-20}{\text{ADC}e^{-4}} &\quad k_2 &= \SI{3.91(18)e-14}{\text{ADC}e^{-3}} \\
  k_3 &= \SI{-2.04(3)e-08}{\text{ADC}e^{-2}} &\quad k_4 &= \SI{3.99(2)e-03}{\text{ADC}e^{-1}} \\
  k_5 &= \SI{-3.71(37)}{\text{ADC}}. &\quad \phantom{f}&\phantom{=10}
\end{align*}

Es wird ebenfalls eine Kalibrationskurve für eine Vorspannung von $\SI{0}{\volt}$ aufgenommen. Diese ist in Abbildung
\ref{fig:kalibration2} für einen großen Bereich mit den anderen Kalibrationskurven dargestellt.

\begin{figure}
  \centering
  \includegraphics{build/kalibration.pdf}
  \caption{Messwerte der Kalibrationsmessung bei einer Vorspannung oberhalb der Depletionsspannung für die Kanäle 20/40/60/80/100 und deren Mittelwert.
  Die Werte liegen alle dicht beieinander und sind in dieser Abbildung ununterscheidbar. Die rote Kurve beschreibt das Ausgleichspolynom vierten Grades.}
  \label{fig:kalibration}
\end{figure}

\begin{figure}
  \centering
  \includegraphics{build/kalibration2.pdf}
  \caption{Messwerte der Kalibrationsmessung bei einer Vorspannung oberhalb der Depletionsspannung für die Kanäle 20/40/60/80/100 und bei $\SI{0}{\volt}$ für Kanal 50.
  Es ist der charge-Bereich, in welchem die Messwerte am meisten voneinander abweichen, dargestellt.}
  \label{fig:kalibration2}
\end{figure}

\subsection{Vermessung der Streifensensoren mittels eines Lasers}

Mit der Option Laser Sync. wird die optimale Verzögerung zwischen Lasersignal und
Chipauslese bestimmt. Da ein Ablesefehler bei der Messung stattfand wird hier (aufgrund des Sollwerts) eine Abschätzung von
\begin{align}
  \text{Laser Delay} = \SI{105}{\nano\second}.
\end{align}
angegeben. In Abbildung \ref{fig:pitch} sind die über die Laserverschiebung gemittelten Signal-Messwerte für jeden Streifen
dargestellt. Es ist erkennbar, dass der Laser über die Streifen 91 und 92 bewegt wurde,
da sich der Signal-Messwert dort im Mittel deutlich von den anderen Werten abhebt.
Die ungemittelten Messwerte für diese beiden Streifen sind in Abbildung \ref{fig:pitch2} dargestellt.
Die Messwerte zu beiden Streifen beschreiben eine Kurve mit zwei herausragenden Maxima.
Das Minimum zwischen diesen beiden Maxima beschreibt jeweils den Fall, dass der Laser genau über dem jeweiligen
Streifen positioniert ist. Um die Pitch zwischen den Streifen zu bestimmen, wird zunächst das Minimum der Kanäle 91 und 92
jeweils durch eine Parabel angenähert. Es ergeben sich für die Funktion
\begin{align}
  \text{Signal}_{\text{min,}i}(x) = u_i x^2 + v_i x + w_i
\end{align}
die Parameter
\begin{align*}
  u_{91} &= \SI{0.1009(22)}{\text{ADC}\meter\tothe{-2}} &\quad v_{91} &= \SI{-19.4(4)}{\text{ADC}\meter\tothe{-1}} \\
  w_{91} &= \SI{927(19)}{\text{ADC}} &\quad \phantom{a} & \phantom{a} \\
  u_{92} &= \SI{0.0894(35)}{\text{ADC}\meter\tothe{-2}} &\quad v_{92} &= \SI{-461(18)e-1}{\text{ADC}\meter\tothe{-1}} \\
  w_{92} &= \SI{5.93(23)e03}{\text{ADC}}. &\quad \phantom{a} & \phantom{a} \\
\end{align*}
Die Pitch zwischen den Streifen (hier 91 und 92) ergibt sich aus der Differenz der Scheitelpunkte der Parabeln gemäß
\begin{align}
  \text{Pitch}_{92-91} = \frac{-v_{92}}{2u_{92}} - \frac{-v_{91}}{2u_{91}} = \SI{162(14)}{\micro\meter}.
\end{align}
Um außerdem die Ausdehnung des Lasers zu bestimmen werden die Signalwerte der relevanten Kanäle 91 und 92 addiert. Es ergibt sich der in Abbildung
\ref{fig:Ausdehnung} dargestellte Verlauf. Die Tatsache, dass kein Plateau in der Pitch der Streifen erkennbar ist, deutet darauf hin, dass der Laser
nicht genügend fokussiert ist. Da die Metallisierung der Kanäle des Silizium-Detektors eine Breite von etwa $\SI{60}{\micro\meter}$ aufweisen und das
Signal bei einer Laserstellung zentral über der Metallisierung auf Null abfällt, kann eine obere Schranke von etwa
\begin{align}
  \text{Ausdehnung} < \SI{60}{\micro\meter}
\end{align}
für die Ausdehnung des Lasers angegeben werden.
%In Abbildung \ref{fig:Ausdehnung} sind die über die Laserverschiebung gemittelten Signal-Messwerte für die Streifen
%88 bis 95 dargestellt. Es wird ein Funktionenfit mit einer Normalverteilung
%\begin{align}
%  P_\text{Laser}(z) = \frac{1}{\sqrt{2 \pi \sigma_\text{ausd}^2}} \text{exp}\left(-\frac{(z-\mu_\text{ausd})^2}{2\sigma_\text{ausd}^2}\right)
%\end{align}
%durchgeführt. Dabei ergibt sich der Mittelwert
%\begin{align}
%  \mu_\text{ausd} = 91.46
%\end{align}
%und die Standardabweichung
%\begin{align}
%  \sigma_\text{ausd} = 0.8303.
%\end{align}
%Es wird angenommen, dass die Ausdehnung des Lasers in etwa der $2\sigma$-Umgebung der Gaußfunktion abzüglich der Distanz, die der Laser bewegt wurde, entspricht.
%Insgesamt folgt daher
%\begin{align}
%  \text{Ausdehnung} = 4\sigma_\text{ausd} \cdot \text{pitch} - \SI{35}{\micro\meter} = \SI{501.96}{\micro\meter}.
%\end{align}

\begin{figure}
  \centering
  \includegraphics{build/pitch.pdf}
  \caption{Gemittelte Signalmesswerte für jeden Kanal. Die Kanäle zu den rot eingezeichneten Messwerten werden für die Bestimmung der pitch benutzt.}
  \label{fig:pitch}
\end{figure}

\begin{figure}
  \centering
  \includegraphics{build/pitch2.pdf}
  \caption{Messwerte für die Kanäle 91 und 92. Die "+"-Messwerte werden für die Parabelfits verwendet.}
  \label{fig:pitch2}
\end{figure}

\begin{figure}
  \centering
  \includegraphics{build/Ausdehnung.pdf}
  \caption{Addierte Messwerte der Kanäle 91 und 92. Aufgetragen ist das summierte Signal gegen die Laserverschiebung.}
  \label{fig:Ausdehnung}
\end{figure}

\subsection{Charge Collection Efficiency Messungen mit dem Laser}

Um einen Zusammenhang zwischen der Charge Collection Efficiency und der Vorspannung zu erhalten,
wird eines der Maxima um Kanal 91 bei verschiedenen Spannungen ausgemessen. Die maximal mögliche Effizienz,
die erreicht werden kann, stellt sich ab der Depletionsspannung ein. Um diese Spannung zu bestimmen muss also
die Stelle, ab der die Signal-Messwerte ein Plateau darstellen, bestimmt werden. Die über alle Events gemittelten
Signal-Messwerte sind in Abbildung \ref{fig:eindringtiefe} gegen die Spannung aufgetragen. Anhand der Messwerte wird ein Funktionenfit mit der unnormierten CCE-Funktion
\begin{align}
  \text{Signal}(U) &= \text{const} \frac{1-\text{exp}\left(\frac{-\SI{300}{\micro\meter}}{a} \sqrt{\frac{U}{U_\text{dep}}}\right)}{1-\text{exp}\left(\frac{-\SI{300}{\micro\meter}}{a}\right)} &\qquad \text{für} \qquad U &< U_\text{dep}\\
  \text{Signal}(U) &= \text{const} &\qquad \text{für} \qquad U &\geq U_\text{dep}
\end{align}
gemäß Gleichung \eqref{eqn:cce} durchgeführt. Dabei ergeben sich die Normierungskonstante
\begin{align}
  \text{const} = \SI{134.6(5)}{ADC},
\end{align}
die Depletionsspannung
\begin{align}
  U_\text{dep} = \SI{73(4)}{\volt}
\end{align}
und die Eindringtiefe des Lasers
\begin{align}
  a_\text{Laser} = \SI{165(15)}{\micro\meter}.
\end{align}

%Es wird eine lineare Ausgleichsrechnung
%mit den Messwerten, die oberhalb der Spannungen $\SI{90}{\volt}$, $\SI{100}{\volt}$ bzw. $\SI{110}{\volt}$ liegen, da dort gemäß Abbildung \ref{fig:ccel}
%der Übergang zum Plateau erwartet wird.
%Für die Geradengleichung
%\begin{align}
%  \text{CCEL-Signal}_i(U) = m_i U + b_i
%\end{align}
%ergeben sich die Koeffizienten
%\begin{align*}
%  m_{\geq90}  &= \SI{0.0083(30)}{\text{ADC}\volt\tothe{-1}} &\quad b_{\geq90}  &= \SI{133.6(5)}{\text{ADC}} \\
%  m_{\geq100} &= \SI{0.0042(24)}{\text{ADC}\volt\tothe{-1}} &\quad b_{\geq100} &= \SI{134.3(4)}{\text{ADC}} \\
%  m_{\geq110} &= \SI{0.0012(19)}{\text{ADC}\volt\tothe{-1}} &\quad b_{\geq110} &= \SI{134.8(3)}{\text{ADC}}.
%\end{align*}
%Der Plateaubeginn wird auf die Spannung mit der geringsten Steigung und geringstem Fehler in der Steigung zurückgeführt.
%Daher ergibt sich in dieser Messung eine Depletionsspannung von etwa
%\begin{align}
%  U_\text{dep,ccel} = \SI{110(5)}{\volt}.
%\end{align}
%Zur Bestimmung der Eindringtiefe des Lasers werden zunächst alle Messwerte auf den Plateaumittelwert normiert
%\begin{align}
%  \text{Plateau}_\text{average} = \SI{135.02}{\text{ADC}},
%\end{align}
%der sich aus dem Mittelwert aller Messwerte oberhalb der Depletionsspannung ergibt.
%In Abbildung \ref{fig:eindringtiefe} ist die resultierende Charge Collection Efficiency für alle aufgenommenen Spannungen unterhalb bzw.
%auf der Depletionsspannung dargestellt. Es wird ein Funktionenfit gemäß Gleichung \eqref{eqn:cce} durchgeführt, woraus sich
%die Eindringtiefe
%\begin{align}
%  a_\text{Laser} = \SI{106.43}{\micro\meter}
%\end{align}
%ergibt.

%\begin{figure}
%  \centering
%  \includegraphics{build/ccel.pdf}
%  \caption{Über alle Events gemittelte Signalmesswerte. Der Plateaubeginn wird bei den Spannungen $\SI{90}{\volt}$, $\SI{100}{\volt}$ oder $\SI{110}{\volt}$ vermutet.
%  Die Ausgleichsgerade für Plateau 110 ist in blau eingezeichnet.}
%  \label{fig:ccel}
%\end{figure}

\begin{figure}
  \centering
  \includegraphics{build/eindringtiefe.pdf}
  \caption{Gemitteltes Signal in Abhängigkeit von der Spannung. Die rote Linie stellt die Fitfunktion der Messwerte dar. Die CCE ergibt sich durch normieren auf den Signalwert des Plateaus.}
  \label{fig:eindringtiefe}
\end{figure}

\subsection{Charge Collection Efficiency Messungen mit der Quelle}
\label{sec:cceq}

Die gegebenen Messwerte werden im ersten Schritt von ADC in $\si{\kilo\electronvolt}$ umgerechnet.
Da zu dem Kalibrationspolynom \eqref{eqn:kalibrationspolynom} keine Umkehrfunktion gebildet werden kann,
wird die Umrechnung durch das Erstellen von Bins durchgeführt. Insgesamt wird die charge im relevanten Bereich
$[0,250000\mathrm{e}]$ in 10000 Bins unterteilt und zu jedem dieser Bins der entsprechende Wert in ADC mittels
des Kalibrationspolynoms \eqref{eqn:kalibrationspolynom} bestimmt. Die Umrechnung von ADC in $\si{\kilo\electronvolt}$
erfolgt dann durch Finden des jeweiligen Bins in ADC, Zuordnen des jeweiligen Bins in charge und als letztes Multiplizieren
mit dem Faktor $\num{3.6e-3}$, da zur Erzeugung eines Elektron-Loch-Paares im Detektor $\SI{3.6}{\electronvolt}$
benötigt werden. Anschließend werden alle Energie-Werte eines Events summiert, woraus sich die jeweilige Clusterenergie ergibt.
Im letzten Schritt werden die Clusterenergien für jede Spannung jeweils über alle Events gemittelt. Die resultierenden
gemittelten Clusterenergien sind in Abbildung \ref{fig:cceq} in Abhängigkeit von der Spannung aufgetragen.
Der Knick wird hier gemäß Abbildung \ref{fig:cceq} bei den Vorspannungen $U = \SI{120}{\volt}$, $U = \SI{130}{\volt}$ und
$U = \SI{140}{\volt}$ vermutet. Es wird erneut eine lineare Ausgleichsrechnung durchgeführt.
Für die Geradengleichung
\begin{align}
  \text{CCEQ-Energie}_j(U) = g_j U + h_j
\end{align}
ergeben sich die Parameter
\begin{align*}
  g_{\geq140} &= \SI{0.018(12)}{\text{ADC}\volt\tothe{-1}} &\quad b_{\geq140} &= \SI{102.1(20)}{\text{ADC}} \\
  g_{\geq130} &=  \SI{0.024(9)}{\text{ADC}\volt\tothe{-1}} &\quad b_{\geq130} &= \SI{101.1(16)}{\text{ADC}} \\
  g_{\geq120} &=  \SI{0.024(7)}{\text{ADC}\volt\tothe{-1}} &\quad b_{\geq120} &= \SI{101.1(12)}{\text{ADC}}.
\end{align*}
Die geringste Steigung ist in diesem Fall dem Plateau 140 zugeordnet, jedoch ist der Fehler hier vergleichsweise groß.
Bei Betrachtung der Messwerte wird daher interpretiert, dass das Plateau etwa bei
\begin{align}
  U_\text{depl,cceq} = \SI{120(5)}{\volt}
\end{align}
beginnt und nicht flach ist, sondern eine leichte Steigung von etwa
\begin{align}
  g_{\geq120} =  \SI{0.024(7)}{\text{ADC}\volt\tothe{-1}}
\end{align}
besitzt.

\begin{figure}
  \centering
  \includegraphics{build/cceq.pdf}
  \caption{In $\si{\kilo\electronvolt}$ umgerechnete Messwerte der CCEQ-Messung. Der Plateaubeginn wird bei den Spannungen $U = \SI{120}{\volt}$, $U = \SI{130}{\volt}$ oder
  $U = \SI{140}{\volt}$ vermutet. Die Ausgleichsgerade für Plateau 120 ist in blau eingezeichnet. Sie besitzt eine leichte Steigung.}
  \label{fig:cceq}
\end{figure}

\subsection{Großer Quellenscan}

Es wird ein RS Run mit $\num{e6}$ Events durchgeführt. Die gemessene relative
Häufigkeit der Anzahl an Clustern pro Event ist in Abbildung \ref{fig:clusterevent} für eine Anzahl von 0 bis 4
Custern aufgetragen, da die Häufigkeit für eine größere Anzahl 0 ist. Die relative Häufigkeit der Anzahl der Kanäle pro Cluster ist
in Abbildung \ref{fig:channelevent} dargestellt. Dabei wird die relative Häufigkeit für $\# \text{Kanäle pro Cluster} > 6$
nicht mit aufgetragen, da sie unter einer Häufigkeit von $h = 0.001$ liegt. Außerdem ist eine Hitmap, also die Anzahl
Ereignisse pro Kanal in Abbildung \ref{fig:hitmap} für alle 128 Kanäle zu sehen.

Das aufgenommene Energiespektrum ist in Abbildung \ref{fig:energieadc} in ADC als
Histogramm dargestellt. Es wird eine Binbreite von $\SI{1}{\text{ADC}}$ gewählt.
Die Messwerte werden wie in Abschnitt \ref{sec:cceq} durch ein Binningverfahren in charge und anschließend
in $\si{\kilo\electronvolt}$ umgewandelt. Die Clusterenergien werden für jedes Event durch Aufsummieren
bestimmt. Erneut werden die resultierenden Daten als normiertes Histogramm dargestellt, siehe Abbildung \ref{fig:energiekeV}. Dabei wird hier eine Binbreite
von $\SI{2.5}{\kilo\electronvolt}$ gewählt. Die Kalibration aus Abschnitt \ref{sec:kalibration} lässt nur in etwa
eine Umrechnung von $0$ ADC bis etwa $269.1$ ADC, bzw. von $0$ bis $250000$ charge, zu. Damit ist noch eine Energie von höchstens
$\SI{900.0}{\kilo\electronvolt}$ erreichbar. Durch Mitteln der Clusterenergien ergibt sich der Mittelwert der deponierten Energie
\begin{align}
  E_\text{average} = \SI{102.87}{\kilo\electronvolt}.
\end{align}
Um den MPV(Most Probable Value) zu bestimmen, wird ein Funktionenfit mit dem Peak aus Abbildung \ref{fig:energiekeV} durchgeführt.
Um die Faltung aus einer Gauß- und einer Landauverteilung zu realisieren, wird eine linear modifizierte Gaußfunktion
\begin{align}
  G_\text{lin}(E) = C \cdot \text{exp}\left(-\frac{\left[(E-\mu)/\si{\kilo\electronvolt}\right]^2}{2\sigma^2} \right) + a \cdot (E-\mu)/\si{\kilo\electronvolt}
\end{align}
verwendet. Dabei ergeben sich die Parameter
\begin{align}
  C &= \num{1.42(1)e-02} & \mu &= \SI{80.3(4)}{\kilo\electronvolt} \\
  \sigma &= \num{21.1(3)} & a &= \num{6.5(6)e-05}.
\end{align}
Der MPV entspricht dem Mittelwert des Funktionenfits
\begin{align}
  \text{MPV} = \mu = \SI{80.3(4)}{\kilo\electronvolt}.
\end{align}

\begin{figure}
  \centering
  \includegraphics{build/cluster_event.pdf}
  \caption{Relative Häufigkeit $h_{\#\text{Cluster}}$ der Anzahl an Clustern pro Event beim großen Quellenscan. Die Häufigkeit für $i>4$ liegt bei $h_{\#\text{Cluster}}=0$ und wird daher nicht mit dargestellt.}
  \label{fig:clusterevent}
\end{figure}

\begin{figure}
  \centering
  \includegraphics{build/channel_event.pdf}
  \caption{Relative Häufigkeit $h_{\#\text{Kanäle}}$ der Anzahl an Kanälen pro Cluster beim großen Quellenscan. Die Häufigkeit bei $i > 6$ ist kleiner als $h = 0.001$ und wird daher nicht mit dargestellt}
  \label{fig:channelevent}
\end{figure}

\begin{figure}
  \centering
  \includegraphics{build/hitmap.pdf}
  \caption{Hitmap für den großen Quellenscan. Es ist die Anzahl an Ereignisse für den jeweiligen Kanal aufgetragen.}
  \label{fig:hitmap}
\end{figure}

\begin{figure}
  \centering
  \includegraphics{build/energiespektrum_adc.pdf}
  \caption{Energiespektrum der Strontium-90 Quelle in ADC, dargestellt in blau. Die rote Linie markiert den maximalen ADC-Wert, der noch zuverlässig mit der Kalibrationskurve umgerechnet werden kann.}
  \label{fig:energieadc}
\end{figure}

\begin{figure}
  \centering
  \includegraphics{build/energiespektrum_keV.pdf}
  \caption{Energiespektrum der Strontium-90 Quelle in $\si{\kilo\electronvolt}$, dargestellt in blau. Die rote Kurve $G_\text{lin}$ beschreibt eine asymmetrische Gaußfunktion, welche die Messwerte am Wahrscheinlichkeits-Maximum des
  Energiespektrums approximieren soll, um den MPV zu bestimmen.}
  \label{fig:energiekeV}
\end{figure}

\begin{table}[h]
  \centering
  \begin{tabular}{S S}
    \toprule
    {$U\:/\:\si{\volt}$} & {$I\:/\:\si{\micro\ampere}$} \\
    \midrule
    199 & 1.20 \\
    190 & 1.20 \\
    180 & 1.19 \\
    170 & 1.19 \\
    160 & 1.18 \\
    150 & 1.18 \\
    140 & 1.17 \\
    130 & 1.16 \\
    120 & 1.15 \\
    110 & 1.14 \\
    100 & 1.12 \\
    90  & 1.11 \\
    80  & 1.09 \\
    70  & 1.06 \\
    60  & 1.04 \\
    50  & 1.00 \\
    40  & 0.96 \\
    30  & 0.91 \\
    20  & 0.86 \\
    10  & 0.82 \\
    0   & 0.78 \\
    \bottomrule
  \end{tabular}
  \caption{Messwerte der Stromspannungskennlinie zur Bestimmung der Depletionsspannung.}
  \label{tab:stromspannungskennlinie}
\end{table}
