\section{Diskussion}
\label{sec:Diskussion}

In Kapitel \ref{sec:stabil} konnte die theoretische maximale Resonatorlänge in beiden Fällen nicht ganz erreicht werden. Im ersten Fall der Versuchsaufbau auf Grund der Länge der optischen Schiene beschränkt und im zweiten Fall wurde die Verstärkung bei der großen Distanz sehr schwach. So lag es an ungenauen Justagen oder Unreinheiten der optischen Elemente, dass die Verstärkung nicht ausreichte.

Die Intensitätsverteilung der TEM-Moden folgt im Rahmen der Fehlerbalken der theoretischen Form.

Auch die Messwerte der Polarisationsmessung haben ungefähr die theoretisch erwartete Form. Die Polarisation durch die Brewsterfenster wie zuvor erklärt ist zu erkennen.

Die Wellenlänge des Lasers konnte mit recht geringer Abweichung bestimmt werden. Der in Kapitel \ref{sec:wellenlaenge} angegebene erwartete Wert ist die Wellenlänge mit maximaler Intensität. Der Laser sendet nun aber auch Licht in anderen Wellenlängen aus. Die Überlagerung dieser verschiedenen Wellenlängen könnte zu einer niedrigeren Wellenlänge führen und so diese Abweichung erklären.

Zwischen den berechneten und per Bandmass gemessenen Resonatorlängen bestehen nur geringe Abweichungen.
