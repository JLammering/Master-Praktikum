\section{Diskussion}
\label{sec:Diskussion}

In Kapitel \ref{sec:stabil} konnte die theoretische maximale Resonatorlänge in beiden Fällen nicht ganz erreicht werden. Im ersten Fall der Versuchsaufbau auf Grund der Länge der optischen Schiene beschränkt und im zweiten Fall wurde die Verstärkung bei der großen Distanz sehr schwach. So lag es an ungenauen Justagen oder Unreinheiten der optischen Elemente, dass die Verstärkung nicht ausreichte.

Die Intensitätsverteilung der TEM-Grundmode folgt im Rahmen der Fehlerbalken der theoretischen Form. Bei der TEM$_{01}$-Mode weicht die Fitkurve im zweiten Bereich stark von den Messwerten ab. Die Messung könnte durch einen verkippten Wolframdraht verfälscht worden sein.

Auch die Messwerte der Polarisationsmessung haben ungefähr die theoretisch erwartete Form. Die Polarisation durch die Brewsterfenster wie zuvor erklärt ist zu erkennen.

Die Wellenlänge des Lasers konnte mit recht geringer Abweichung bestimmt werden. In Kapitel \ref{sec:wellenlaenge} ist der erwartete Wert angegeben. Die hauptsächliche Fehlerquelle ist hier das ungenaue Ablesen der Intensitätsmaxima.

Zwischen den berechneten und per Bandmass gemessenen Resonatorlängen bestehen nur geringe Abweichungen. Die Möglichkeit einer Schwebung der verschiedenen Moden konnte durch eine Abschätzung gezeigt werden.
