\section{Auswertung}
\label{sec:Auswertung}

\subsection{Überprüfung der Stabilitätsbedingung}

Die Stabilitätsbedingung aus Gleichung \eqref{eqn:1} wird für zwei Kombinationen der Spiegel überprüft.
Als Auskopplungsspiegel wird ein Spiegel genutzt bei dem die reflektierende Fläche einen Krümmungsradius von $r = \SI{1400}{\milli\meter}$ besitzt. Der stark refelektierende Spiegel hat erst ebenfalls einen Krümmungsradius von $r = \SI{1400}{\milli\meter}$ und wird dann in der zweiten überprüften Kombination durch einen flachen Spiegel ersetzt.
Die theoretisch erreichbare maximale Länge $L$ des Resonators kann in Abbildung \ref{fig:stabil} abgelesen werden. Wie auch in Formel \eqref{eqn:1} zu erkennen ist, liegt diese bei zwei krummen Spiegeln bei $r_1+r_2$ und bei der zweiten Kombination bei $r_1$.

\begin{figure}
  \centering
  \includegraphics[height=9cm]{build/stabil.pdf}
  \caption{Die theoretisch erreichbaren Verstärkungsfaktoren der beiden Spiegelkombinationen. }
  \label{fig:stabil}
\end{figure}

Die experimentell erreichten maximalen Resonatorlängen sind in Tabelle \ref{tab:stabil} zu sehen.

\begin{table}[h]
  \centering
  \begin{tabular}{l l S}
    \toprule
    {HR-Spiegel}& {OC-Spiegel} & {$L_\text{max}\:/\:\si{\centi\meter}$}\\
    \midrule
    $r = \SI{1400}{\milli\meter}$/flat & $r = \SI{1400}{\milli\meter}$/flat & 215.4\\
    flat/flat & $r = \SI{1400}{\milli\meter}$/flat & 126.5\\
    \bottomrule
  \end{tabular}
  \caption{Die erreichten maximalen Resonatorlängen.}
  \label{tab:stabil}
\end{table}

\subsection{Vermessung der TEM-Moden}

\paragraph{TEM$_{00}$ Grundmode}

Die gemessene Intensität der TEM$_{00}$ Grundmode in Abhängigkeit der Position der Photodiode ist in Tabelle \ref{tab:grundmode} eingetragen.

\begin{table}[h]
  \centering
  \begin{tabular}{S S}
    \toprule
    {$x\:/\:\si{\milli\meter}$} & {$I\:/\:\si{\micro\ampere}$}\\
    \midrule
    28 & 0.22\\
    25 & 0.512\\
    22 & 0.96\\
    19 & 1.593\\
    16 & 2.375\\
    13 & 2.946\\
    10 & 2.971\\
    7 & 2.489\\
    4 & 1.78\\
    1 & 1.152\\
    -2 & 0.667\\
    -5 & 0.386\\
    -8 & 0.181\\
    \bottomrule
  \end{tabular}
  \caption{Messwerte der Intensitätsverteilung der TEM$_{00}$ Grundmode mit den Unsicherheiten $\sigma_x = \SI{0.5}{\milli\meter}$ und $\sigma_I = \SI{0.01}{\micro\ampere}$.}
  \label{tab:grundmode}
\end{table}

An diese Messwerte wurde die Funktion
\begin{align}
  I = I_0*\exp((-2*(x+x_0)^2)/(\omega^2))
\end{align}
gefittet. Es ergeben sich die Werte:
\begin{align}
  I_0 = \SI{1.25(3)}{\micro\ampere} \quad x_0 = \SI{-11.1(1)}{\milli\meter} \quad \omega = \SI{14.7(2)}{\milli\meter}.
\end{align}
Die Messwerte und die Fitfunktion sind in Abbildung \ref{fig:grundmode} zu sehen.

\begin{figure}
  \centering
  \includegraphics[height=5cm]{build/grundmode.pdf}
  \caption{Messwerte und Fitfunktion für die TEM$_{00}$ Grundmode.}
  \label{fig:grundmode}
\end{figure}







\begin{figure}
  \centering
  \includegraphics[height=5cm]{build/plotElement.pdf}
  \caption{Plot.}
  \label{fig:plot}
\end{figure}

Tabelle für copy and paste:
\begin{table}[h]
  \centering
  \begin{tabular}{S S}
    \toprule
    {$k$} & {$U\:/\:\si{\milli\volt}$}\\
    \midrule
    1 & 637.2\\
    3 & 212.4\\
    5 & 127.4\\
    7 & 91.03\\
    9 & 70.8\\
    \bottomrule
  \end{tabular}
  \caption{Amplituden Rechteckspannung.}
  \label{tab:rechtampl}
\end{table}
