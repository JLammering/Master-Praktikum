\section{Diskussion}
\label{sec:Diskussion}

\subsection{Magnetfeldkalibrierung}

Im ersten Versuchsteil wurde das Magnetfeld des Elektromagneten
in Abhängigkeit von der Stromstärke ausgemessen. Es wurde ein relativ großer Fehler von
$\SI{50}{\milli\tesla}$ angenommen, der sich dadurch begründen lässt, dass es
durch die manuelle Messung mit der Hallsonde zu Winkelabweichungen von der waagerechten Position kommt
und somit Schwankungen der gemessenen Flussdichte resultieren. Der angenommene Fehler
hat auch ein sichtliches Fehlerintervall in dem resultierenden Abzissenabschnitt der Ausgleichsgeraden
\begin{align}
  b &= \SI{42(19)}{\milli\tesla}
\end{align}
zur Folge. An dem Graphen in Abbildung \ref{fig:kalib} und an dem Fehlerintervall der resultierenden Steigung
\begin{align}
  m &= \SI{5.39(16)e01}{\milli\tesla\per\ampere}
\end{align}
ist jedoch erkennbar, dass die Gerade den Messwerten sehr gut folgt. Betrachtet man den Verlauf der
Messwerte sehr genau, so kann ein leicht S-förmiger Verlauf beobachtet werden. Dieser ist mit
hoher Sicherheit Hysterese-Effekten im Elektromagneten zuzuschreiben, hat aber keine
sichtlichen Konsequenzen auf die Ausgleichsrechnung.

\subsection{Bestimmung der Landéfaktoren}

In Tabelle \ref{tab:landefaktoren} sind Mess- und Theoriewerte der Landéschen Faktoren für
die ausgemessenen Spektrallinien der Cd-Lampe eingetragen.
Bei der $\sigma$-Messung handelt es sich bei dem angegebenen Theoriewert
um den Mittelwert von den tatsächlich im Spektrum vorkommenden Landéfaktoren von
$g_{J,\sigma,\text{theorie,I}} = 1.5$ und $g_{J,\sigma,\text{theorie,II}} = 2$,
da die beiden Spektrallinien wie schon erwähnt in der Messung zusammenfallen.

\begin{table}[H]
  \centering
  \caption{Messwerte $g_J$, Theoriewerte $g_{J,\text{theorie}}$ und relative Abweichungen der Landéfaktoren für
  die rote und blaue Spektrallinie der Cd-Lampe}
  \begin{tabular}{c S[table-format=1.2] S[table-format=1.2] @{${}\pm{}$} S[table-format=1.2] S[table-format=1.1]}
    \toprule
    {Linie} & {$g_{J,\text{theorie}}$} & \multicolumn{2}{c}{$g_J$} & {$\Delta_\text{rel}/\si{\percent}$}\\
    \midrule
    {rot}         & 1.0  & 0.96 & 0.07 & 4.0 \\
    \midrule
    {blau, $\pi$}    & 0.5  & 0.50 & 0.04 & 0.0 \\
    {blau, $\sigma$} & 1.75 & 1.60 & 0.14 & 8.6 \\
    \bottomrule
    \label{tab:landefaktoren}
  \end{tabular}
\end{table}

Bei der roten Spektrallinie und der blauen $\pi$-Spektrallinie konnten die Theoriewerte
mit sehr geringen Abweichungen von nur $\SI{4}{\percent}$ und $\SI{0}{\percent}$(!)
bestätigt werden. Hier zeigt sich, dass die Messmethode mit der Lummer-Ghercke-Platte
gut funktioniert. Größere Abweichungen von $\SI{8.6}{\percent}$ treten bei dem
Landéfaktor der blauen $\sigma$-Spektrallinie(n) auf. Diese Abweichung lässt sich unter anderem
dadurch erklären, dass eben durch das Zusammenfallen zweier Spektrallinien aufgrund von
nicht ausreichendem Auflösungsvermögen auch die Messungenauigkeit des Mittelwerts ansteigt.
Für eng beeinander liegende Spektrallinien scheint der Versuchsaufbau daher weniger geeignet zu sein.
