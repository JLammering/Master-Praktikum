\appendix

\section{Tabellen}

\begin{table}[H]
  \centering
  \caption{Magnetfeldmesswerte $B$ in Abhängigkeit von der Stromstärke $I$.}
  \begin{tabular}{S[table-format=2.0] S[table-format=4.0]}
    \toprule
    {$I/\si{\ampere}$} & {$B/\si{\tesla}$}\\
    \midrule
    0  & 6    \\
    1  & 66   \\
    3  & 191  \\
    5  & 316  \\
    7  & 433  \\
    9  & 559  \\
    11 & 672  \\
    13 & 782  \\
    15 & 881  \\
    17 & 973  \\
    19 & 1038 \\
    21 & 1102 \\
    \bottomrule
  \end{tabular}
  \label{tab:kalib}
\end{table}

\begin{table}[H]
  \centering
  \caption{Abstandsmesswerte $\Delta s$ der roten Spektrallinie in Abhängigkeit von der Ordnung $n-n_0$.
  Dabei bezieht sich $n-n_0$ jeweils auf die linke Linie der beiden am Abstand beteiligten Linien.}
  \begin{tabular}{S[table-format=2.0] S[table-format=3.0]}
    \toprule
    {$n-n_0$} & {$\Delta s/\text{LE}$}\\
    \midrule
    0  & 309 \\
    1  & 265 \\
    2  & 233 \\
    3  & 212 \\
    4  & 194 \\
    5  & 181 \\
    6  & 169 \\
    7  & 161 \\
    8  & 153 \\
    9  & 147 \\
    10 & 143 \\
    11 & 135 \\
    \bottomrule
  \end{tabular}
  \label{tab:redB0}
\end{table}

\begin{table}[H]
  \centering
  \caption{Abstandsmesswerte $\delta s$ der aufgespaltenen roten Spektrallinie in Abhängigkeit von der Ordnung $n-n_0$.}
  \begin{tabular}{S[table-format=2.0] S[table-format=3.0]}
    \toprule
    {$n-n_0$} & {$\delta s/\text{LE}$}\\
    \midrule
    0  & 151 \\
    1  & 129 \\
    2  & 114 \\
    3  & 105 \\
    4  & 91  \\
    5  & 85  \\
    6  & 79  \\
    7  & 76  \\
    8  & 76  \\
    9  & 71  \\
    10 & 65  \\
    11 & 67  \\
    \bottomrule
  \end{tabular}
  \label{tab:redB}
\end{table}

\begin{table}[H]
  \centering
  \caption{Quotienten $\delta s/\Delta s$ der roten Spektrallinie in Abhängigkeit von der Ordnung $n-n_0$.}
  \begin{tabular}{S[table-format=2.0] S[table-format=1.2] @{${}\pm{}$} S[table-format=1.2]}
    \toprule
    {$n-n_0$} & \multicolumn{2}{c}{$\delta s/\Delta s$}\\
    \midrule
    0  & 0.49 & 0.05 \\
    1  & 0.49 & 0.06 \\
    2  & 0.49 & 0.07 \\
    3  & 0.50 & 0.07 \\
    4  & 0.47 & 0.08 \\
    5  & 0.47 & 0.09 \\
    6  & 0.47 & 0.09 \\
    7  & 0.47 & 0.10 \\
    8  & 0.50 & 0.10 \\
    9  & 0.48 & 0.11 \\
    10 & 0.45 & 0.11 \\
    11 & 0.50 & 0.12 \\
    \bottomrule
  \end{tabular}
  \label{tab:redquot}
\end{table}

\begin{table}[H]
  \centering
  \caption{Abstandsmesswerte $\Delta s$ der blauen Spektrallinie in Abhängigkeit von der Ordnung $n-n_0$.
  Dabei bezieht sich $n-n_0$ jeweils auf die linke Linie der beiden am Abstand beteiligten Linien.}
  \begin{tabular}{S[table-format=2.0] S[table-format=3.0]}
    \toprule
    {$n-n_0$} & {$\Delta s/\text{LE}$}\\
    \midrule
    0  & 317 \\
    1  & 259 \\
    2  & 226 \\
    3  & 202 \\
    4  & 180 \\
    5  & 174 \\
    6  & 158 \\
    7  & 150 \\
    8  & 143 \\
    9  & 138 \\
    10 & 125 \\
    \bottomrule
  \end{tabular}
  \label{tab:blueB0}
\end{table}

\begin{table}[H]
  \centering
  \caption{Abstandsmesswerte $\delta s_\pi$ und $\delta s_\sigma$ der blauen aufgespaltenen
  Spektrallinien in Abhängigkeit von der Ordnung $n-n_0$.}
  \begin{tabular}{S[table-format=2.0] S[table-format=3.0] S[table-format=3.0]}
    \toprule
    {$n-n_0$} & {$\delta s_\pi/\text{LE}$} & {$\delta s_\sigma/\text{LE}$}\\
    \midrule
    0  & 135 & 143 \\
    1  & 117 & 127 \\
    2  & 109 & 105 \\
    3  & 85  & 89  \\
    4  & 78  & 83  \\
    5  & 77  & 76  \\
    6  & 71  & 75  \\
    7  & 61  & 70  \\
    8  & 55  & 65  \\
    9  & 54  & 62  \\
    10 & 52  & 60  \\
    \bottomrule
  \end{tabular}
  \label{tab:blueBpiandsigma}
\end{table}

\begin{table}[H]
  \centering
  \caption{Quotienten $\delta s/\Delta s$ der blauen $\pi$- und $\sigma$-Spektrallinie in Abhängigkeit von der Ordnung $n-n_0$.}
  \begin{tabular}{S[table-format=2.0] S[table-format=1.2] @{${}\pm{}$} S[table-format=1.2] S[table-format=1.2] @{${}\pm{}$} S[table-format=1.2]}
    \toprule
    {$n-n_0$} & \multicolumn{2}{c}{$(\delta s/\Delta s)_\pi$} & \multicolumn{2}{c}{$(\delta s/\Delta s)_\sigma$}\\
    \midrule
    0  & 0.43 & 0.05 & 0.45 & 0.05 \\
    1  & 0.45 & 0.06 & 0.49 & 0.06 \\
    2  & 0.48 & 0.07 & 0.46 & 0.07 \\
    3  & 0.42 & 0.08 & 0.44 & 0.08 \\
    4  & 0.43 & 0.09 & 0.46 & 0.09 \\
    5  & 0.44 & 0.09 & 0.44 & 0.09 \\
    6  & 0.45 & 0.10 & 0.47 & 0.10 \\
    7  & 0.41 & 0.10 & 0.47 & 0.10 \\
    8  & 0.38 & 0.11 & 0.45 & 0.11 \\
    9  & 0.39 & 0.11 & 0.45 & 0.11 \\
    10 & 0.42 & 0.12 & 0.48 & 0.13 \\
    \bottomrule
  \end{tabular}
  \label{tab:bluequot}
\end{table}

\section{Kopie der Originaldaten}
