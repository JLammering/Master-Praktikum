\section{Diskussion}
\label{sec:Diskussion}

\subsection{Amplitudenmodulierte Schwingung mit Ringmodulator}

Die Amplitudenmodulation mit dem Ringmodulator liefert die erwarteten Ergebnisse. Die Amplitudenvariation der entstandenen Schwebung hat sehr genau dieselbe Frequenz  wie die Modulationsfrequenz. Auch das Frequenzspektrum und die geringe Abweichung zu den erwarteten Frequenzen zeigt, dass der Ringmodulator gut zur Modulation geeignet ist. Außerdem erkennt man in den Abbildungen des Frequenzspektrum, dass die Leistung der Trägerfrequenz unterdrückt ist.

\subsection{Amplitudenmodulierte Schwingung mit Diode}

Die beiden auf verschiedenen Wegen ausgerechneten Werte für den Modulationsgrad sind bei diesem Aufgabenteil leider sehr unterschiedlich. Da der Modulationsgrad auch ein Maß dafür ist, wie stark die Amplituden variieren erscheint der Wert $m = \num{0.2959(4)}$ nach Betrachtung der Abbildung \ref{fig:amplModDiode} als sinnvoller. Also muss bei der Berechnung aus der Intensität der Frequenzpeaks ein Fehler passiert sein.

\subsection{Frequenzmodulierte Schwingung}

Die Frequenzmodulation liefert das erwartete Oszilloskopbild. Man erkennt, dass die Frequenz variiert und deshalb die Kurven der Funktionenschar bei der selben Phase zeitlich weiter vorangeschritten sein können.
Der berechnete Wert für den Modulationsgrad ist erneut sehr klein und die Herleitung der entsprechenden Formel vermutlich falsch.

\subsection{Demodulation mithilfe eines Ringmodulators -- Überprüfung der Spannung}

Die Messwerte aus Abbildung \ref{fig:plotphase} zeigen nicht wie in \cite{anleitung} gefordert, dass die Spannung proportional zum Kosinus der Phasenverschiebung ist. Es zeigt sich aber ein kosinusförmiger Zusammenhang zwischen diesen Größen. Wenn diese Forderung gemeint ist, dann ist konnte ein gutes Ergebnis erzielt werden.


\subsection{Demodulation mithilfe eines Ringmodulators -- Überprüfung des Oszilloskopbild}

Es ist wie gefordert zu erkennen, dass die Modulationsfrequenz nur leicht verzerrt wiedergewonnen werden kann. Außerdem ist zu erkennen, dass die Amplitude durch die Bauteile stark abgeschwächt ist.

\subsection{Demodulation mithilfe einer Gleichrichterdiode}

In Abbildung \ref{fig:gnachA} ist die Wirkung der Diode zu sehen, da sie die Spannung größtenteils nur in Vorzugsrichtung durchlässt und nur wenig unter der Nulllinie auf dem Oszilloskop anzeigt. 


\subsection{Demodulation einer frequenzmodulierten Spannung}

Im Bild nach dem Schwingkreis erkennt man, dass die Spannung schon von einer frequenzmodulierten Spannung in eine Amplitudenmodulierte überführt wurde.
Nach der Diode ist ein ähnliches Bild wie in Abbildung \ref{fig:gnachA} zu sehen. Die ausgehende Spannung hat eine nicht symmetrische Form und eine kleine Amplitude. Dadurch hat auch die Kurve aus Abbildung \ref{fig:hnachTiefpass} eine kleine Amplitude und ist stark verzerrt. Mit diesen Bauteilen ist die Demodulation einer frequenzmodulierten Spannung also nicht optimal.
