\section{Auswertung}
\label{sec:Auswertung}

\subsection{Amplitudenmodulierte Schwingung mit Ringmodulator}

Mit Hilfe der Schaltung aus Abbildung \ref{fig:???} wurde eine Amplitudenmodulierte Schwingung erzeugt.
Diese so entstandene Schwebung ist in Abbildung \ref{fig:amplModOszi} zu sehen.

\begin{figure}
  \centering
  \includegraphics{/path/to/figure}
  \caption{}
  \label{}
\end{figure}


\begin{figure}
  \centering
  \includegraphics{plotphase.pdf}
  \caption{Plot.}
  \label{fig:plot}
\end{figure}

Tabelle für copy and paste:
\begin{table}[h]
  \centering
  \begin{tabular}{S S}
    \toprule
    {$k$} & {$U\:/\:\si{\milli\volt}$}\\
    \midrule
    1 & 637.2\\
    3 & 212.4\\
    5 & 127.4\\
    7 & 91.03\\
    9 & 70.8\\
    \bottomrule
  \end{tabular}
  \caption{Amplituden Rechteckspannung.}
  \label{tab:rechtampl}
\end{table}
