\section{Auswertung}
\label{sec:Auswertung}

\listoftodos

\subsection{Amplitudenmodulierte Schwingung mit Ringmodulator}

Mit Hilfe der Schaltung aus Abbildung \ref{fig:expab} wird eine Amplitudenmodulierte Schwingung erzeugt.
Diese so entstandene Schwebung ist in Abbildung \ref{fig:amplModOszi} zu sehen.

\begin{figure}[h]
  \centering
  \includegraphics[width=.9\textwidth]{Oszi_Pics/amplModRing.png}
  \caption{Amplitudenmodulierte Schwingung(gelb) und Modulationsschwingung(grün) erzeugt mit Ringmodulator.}
  \label{fig:amplModOszi}
\end{figure}

Die mit dem Oszilloskop außerdem ausgemessenen Werte für die Frequenzen $f$ und Amplituden $U_\text{peak to peak}$ der Modulationsspannung $M$ und der Trägerspannung $T$ sind:

\begin{align*}
  f_\text{M} &= \SI{43.8(5)}{\kilo\hertz} & U_\text{M, ptp} &= \SI{95(1)}{\milli\volt}\\
  f_\text{T} &= \SI{970(1)}{\kilo\hertz} & U_\text{T, ptp} &= \SI{540(1)}{\milli\volt}.
\end{align*}

Die mit dem Frequenzspektrometer aufgenommenen Werte sind in den Bildern \ref{fig:b1}, \ref{fig:b2} und \ref{fig:b3} zu sehen.

\begin{figure}[h]
  \includegraphics[width=.9\textwidth]{Spektrum_Pics/b1.jpg}
  \caption{Spektrum der mit dem Ringmodulator amplitudenmodulierten Schwingung mit Markierung vom Peak $f_\text{T} - f_\text{M}$}
  \label{fig:b1}
\end{figure}


Die aus dem Frequenzspektrum abgelesenen Frequenzen und Leistungspegel für die drei größten Peaks sind:
\begin{align*}
  f_1 &= \SI{929.3}{\kilo\hertz} & f_2 &= \SI{973.1}{\kilo\hertz} & f_3 &= \SI{1016.6}{\kilo\hertz}.\\
  L_{\text{P}, 1} &= \SI{-19.04}{dBm} & L_{\text{P}, 2} &= \SI{-47.6}{dBm} & L_{\text{P}, 3} &= \SI{-19.02}{dBm}
\end{align*}
Die Abweichungen zu den erwarteten Werten betragen:
\begin{align*}
  \frac{|(f_\text{T} - f_\text{M}) - f_1|}{f_\text{T} - f_\text{M}} &= \SI{0.3(1)}{\percent}\\
  \frac{|f_\text{T} - f_2|}{f_\text{T}} &= \SI{0.3(1)}{\percent}\\
  \frac{|(f_\text{T} + f_\text{M}) - f_3|}{f_\text{T} + f_\text{M}} &= \SI{0.3(1)}{\percent}.
\end{align*}

\subsection{Amplitudenmodulierte Schwingung mit Diode}
\label{sec:amplModDiode}

Nach Schaltung aus Abbildung \ref{fig:expc} wird hier der allgemeine Fall der Amplitudenodulation mit einer Diode gezeigt.

\begin{figure}[h]
  \centering
  \includegraphics[width=.9\textwidth]{Oszi_Pics/amplModDiode.png}
  \caption{Amplitudenmodulierte Schwingung erzeugt mit einer Diode.}
  \label{fig:amplModDiode}
\end{figure}

Die so entstandene Schwebung ist in Abbildung \ref{fig:amplModDiode} zu sehen.
Außerdem wurden folgende Kennzahlen der modulierten Spannung ausgemessenen:
die Spannungsdifferenz von der Nulllinie im Oszilloskop bis zum höchsten Wert $U_\text{max}$, die Differenz zwischen der höchsten und der niedrigsten Amplitude $U_\text{diff}$ sowie die Spannungsdifferenz von der Nulllinie bis zur unteren Kante also der Fehler der Diode $U_\text{fehler}$.
Diese Werte betragen:
\begin{align*}
  U_\text{max} &= \SI{43(1)}{\milli\volt} & U_\text{diff} &= \SI{20(1)}{\milli\volt}  & U_\text{fehler} &= \SI{14(1)}{\milli\volt}
\end{align*}
Aus Bild \ref{fig:amplmodskizze} wird die Formel
\begin{align}
  m &= \frac{U_\text{max}}{U_\text{max} - \frac{U_\text{diff}}{2}} - 1\\
  \intertext{hergeleitet.}
  \intertext{Das ergibt für den Modulationsgrad:}
  m &= \num{0.30(2)}.
\end{align}

Nun wird der Modulationsgrad aus der Frequenzspektrumsmessung abgelesenen.
\begin{figure}[h]
  \centering
  \includegraphics[width=.9\textwidth]{Spektrum_Pics/c.jpg}
  \caption{Spektrum der mit der Diode amplitudenmodulierten Schwingung mit einer Peaktable.}
  \label{fig:c}
\end{figure}
Das Bild mit dem Spektrum der Frequenzen ist in Abbildung \ref{fig:c} zu sehen.
Dabei werden die drei höchsten Peaks den Frequenzen $f_\text{T} - f_\text{M}$, $f_\text{T}$ und $f_\text{T} + f_\text{M}$ zugeordnet.
Die Peaktable zeigt die Werte für die Leistungspegel $L_\text{P}$:
\begin{align*}
  L_\text{P, links} &= \SI{-36.1(1)}{dBm} & L_\text{P, mitte} &= \SI{-20.55(10)}{dBm} & L_\text{P, rechts} &= \SI{-35.88(10)}{dBm}.
\end{align*}
Diese Werte werden nun mit Gleichung \eqref{eqn:dBmTomW} aus \cite{leistungspegel} umgerechnet in Werte für die Leistung der Frequenzen.
\begin{align}
  P(\si{\milli\watt}) =
   10^{\frac{L_\text{P}(\si{dBm})}{10}} \SI{1}{\milli\watt} \label{eqn:dBmTomW}
\end{align}

Diese betragen:
\begin{align*}
  P_\text{links} &= \SI{0.245(6)}{\micro\watt} & P_\text{mitte} &= \SI{8.81(2)}{\micro\watt} & P_\text{rechts} &= \SI{0.258(6)}{\micro\watt}.
\end{align*}
Diese Werte für die Leistung werden mit
\begin{align}
  U &= \sqrt{P R}
\end{align}
in die entsprechenden Spannungen umgerechnet. Dabei ist der Widerstand $R$ konstant.
Aus Bild \ref{fig:freqspektrum} wird nun die Formel zur Bestimmung des Modulationsgrad aus einem Frequenzspektrum hergeleitet:
\begin{align*}
  m &= \frac{2 U_\text{lr}}{U_\text{mitte}}.
\end{align*}

Hierbei ist $U_\text{lr}$ der Mittelwert des linken und des rechten Peaks der drei Peaks mit der höchsten Leistung.
So ergibt sich für $m$:
\begin{align*}
  m &= \num{0.338(5)}.
\end{align*}

\subsection{Frequenzmodulierte Schwingung}

Es wurde eine frequenzmodulierte Schwingung erzeugt mit den Frequenzen:
\begin{align*}
  f_\text{M} &= \SI{211.5(4)}{\kilo\hertz} \quad\text{und}\quad f_\text{T} &= \SI{973(2)}{\kilo\hertz}.
\end{align*}
\begin{figure}[h]
  \centering
  \includegraphics[width=.9\textwidth]{Oszi_Pics/freqModRing.png}
  \caption{Oszilloskopbild der frequenzmodulierten Schwingung.}
  \label{fig:freqModRing}
\end{figure}
\begin{figure}[h]
  \centering
  \includegraphics[width=.9\textwidth]{Oszi_Pics/freqModZoom.png}
  \caption{Vergrößertes Oszilloskopbild der frequenzmodulierten Schwingung.}
  \label{fig:freqModZoom}
\end{figure}
In Abbildung \ref{fig:freqModRing} ist die in X-Richtung verschmierte Sinuskurve der nach Schaltung \ref{fig:freqmodschaltung} frequenzmodulierten Schwingung zu sehen. Außerdem ist in Abbildung \ref{fig:freqModZoom} eine vergrößerte Version der Schwingung zu sehen. Daraus wird die Zeitdifferenz bei maximaler Frequenzvariation abgelesenen:
\begin{align*}
  t_2 - t_1 &= \SI{288(5)}{\nano\second}.\\
\intertext{ergibt sich eingesetzt in Formel \eqref{eqn:mfuerfreqmod} für den Modulationsgrad}
  m &= \num{0.0416(7)}.
\end{align*}\todo{mit neuer Formel ausrechnen}

Der Modulationsgrad wird erneut auch mit dem Frequenzspektrum bestimmt. Das Bild des Spektrometers ist in Abbildung \ref{fig:freqModSpek} zu sehen.
\begin{figure}[h]
  \centering
  \includegraphics[width=.9\textwidth]{Spektrum_Pics/d.jpg}
  \caption{Frequenzspektrum der frequenzmodulierten Schwingung.}
  \label{fig:freqModSpek}
\end{figure}
Dort lassen sich für die Leistungspegel der drei höchsten Peaks die Werte mit entsprechenden Unsicherheiten ablesen:
\begin{align*}
  L_\text{P, links} &= \SI{-22.0(1)}{dBm} & L_\text{P, mitte} &= \SI{-9.9(1)}{dBm} & L_\text{P, rechts} &= \SI{-22.2(1)}{dBm}.
\end{align*}
Die Rechnung aus Kapitel \ref{sec:amplModDiode} wird nun hier mit den veränderten Amplituden für die beiden äußeren Peaks aus Gleichung \eqref{eqn:freqmod2} modifiziert. So ergibt sich für den Modulationsgrad hier:
\begin{align*}
  m &= \frac{2 U_\text{lr}}{U_\text{mitte}} \frac{f_\text{M}}{f_\text{T}}\\
  &= \num{0.1067(15)}.
\end{align*}
Damit kann die Forderung an die Schmalband-Frequenzmoduation überprüft werden. Mit
\begin{align*}
  m \frac{f_T}{f_M} = \num{0.491(7)} << 1
\end{align*}
ist diese erfüllt.

\subsection{Demodulation mithilfe eines Ringmodulators}

Mit der Schaltung aus Abbildung \ref{fig:expef} wird eine Messreihe von verschiedenen Frequenzen der Trägerspannung $f_\text{T}$ und der am Ausgang X anliegenden Gleichspannung $U$ aufgenommen.

Die beiden Spannungen, die an den Eingängen L und R anliegen, sind durch einen Phasenschieber mit einer Phasenverzögerung von $\Delta t = \SI{250}{\nano\second}$ zueinander verschoben.
Die Phasenverschiebung wird berechnet mit:
\begin{align}
  \Delta \phi = 2\pi \Delta t  f.
\end{align}
In Tabelle \ref{tab:phase} sind die aufgenommenen Werte sowie die berechneten Phasenverschiebungen zwischen den beiden Spannungen eingetragen.

\begin{table}[h]
  \centering
  \begin{tabular}{S[table-format=1.3]
     S[table-format=2.3]
     S[table-format=1.2]
     }
    \toprule
    {$f\:/\:\si{\mega\hertz}$} & {$U\:/\:\si{\volt}$} & {$\Delta \phi$}\\
    \midrule
    0.102 & -0.155 & 0.16\\
    0.299 & -0.136 & 0.47\\
    0.599 & -0.075 & 0.94\\
    1.0 & 0.012 & 1.57\\
    1.33 & 0.077 & 2.09\\
    1.675 & 0.136 & 2.63\\
    2.0 & 0.175 & 3.14\\
    2.515 & 0.108 & 3.95\\
    3.015 & 0.004 & 4.74\\
    3.5 & -0.093 & 5.5\\
    3.985 & -0.166 & 6.26\\
    4.495 & -0.126 & 7.06\\
    4.995 & -0.028 & 7.85\\
    5.5 & 0.075 & 8.64\\
    5.98 & 0.153 & 9.39\\
    \bottomrule
  \end{tabular}
  \caption{Die Werte für die aufgenommenen Frequenzen und Spannungen sowie die berechnete Phasenverschiebung. Die  Unsicherheiten sind $\Delta f = \SI{0.005}{\mega\hertz}$, $\Delta U = \SI{0.001}{\volt}$ und $\Delta \phi = \num{0.001}$.}
  \label{tab:phase}
\end{table}

In der Abbildung \ref{fig:plotphase} ist die Spannung gegen den Kosinus der Phase aufgetragen.
\todo{Fit in Phasenplot}

\begin{figure}
  \centering
  \includegraphics[width=.9\textwidth]{build/plotphase.pdf}
  \caption{Diagramm der Spannung in Abhängigkeit vom Kosinus der Phase. Die Unsicherheiten sind aufgrund ihrer geringen Größe nicht zu sehen.}
  \label{fig:plotphase}
\end{figure}

Mit der Schaltung nach Abbildung \ref{fig:ampldemodschaltung1} wird eine amplitudenmodulierte Spannung demoduliert und auf dem Oszilloskop dargestellt. Das ergebende Bild ist in Abbildung \ref{fig:demodRing} zu sehen.

\begin{figure}[h]
  \centering
  \includegraphics[width=.9\textwidth]{Oszi_Pics/demodRing.png}
  \caption{Oszilloskopbild der demodulierten Spannung(grün) sowie der Original-Modulationsspannung(gelb).}
  \label{fig:demodRing}
\end{figure}

\subsection{Demodulation mithilfe einer Gleichrichterdiode}

Nun wird mithilfe einer Gleichrichterdiode nach Schaltung aus Abbildung \ref{fig:ampldemodschaltung2} eine amplitudenmodulierte Spannung demoduliert.
Das aufgenommene Bild an der Stelle A ist in Abbildung \ref{fig:gnachA} zu finden und das nach dem Tiefpass in Abbildung \ref{fig:gnachTiefpass}.
\begin{figure}[h]
  \centering
  \includegraphics[width=.9\textwidth]{Oszi_Pics/gnachA.png}
  \caption{Oszilloskopbild der demodulierten Spannung(grün) sowie der Original-Modulationsspannung(gelb) an der Stelle A.}
  \label{fig:gnachA}
\end{figure}
\begin{figure}[h]
  \centering
  \includegraphics[width=.9\textwidth]{Oszi_Pics/gnachTiefpass.png}
  \caption{Oszilloskopbild der demodulierten Spannung(grün) sowie der Original-Modulationsspannung(gelb) nach dem Tiefpass.}
  \label{fig:gnachTiefpass}
\end{figure}
Zu erkennnen ist, dass die demodulierte Spannung die doppelte Frequenz der Original-Modulationsspannung hat.

\subsection{Demodulation einer frequenzmodulierten Spannung}

Zum Abschluss wird erneut eine frequenzmodulierte Spannung mit Abbildung \ref{fig:freqmodschaltung} erzeugt und mit der Schaltung nach Abbildung \ref{fig:flankendemodulator} demoduliert. Dann werden 3 Bilder nach den verschiedenen Bauteilen aufgenommen.
Diese sind in den Abbildungen \ref{fig:hnachSchwingkreis}, \ref{fig:hnachDiode} und \ref{fig:hnachTiefpass} zu sehen.
\begin{figure}[h]
  \centering
  \includegraphics[width=.9\textwidth]{Oszi_Pics/hnachSchwingkreis.png}
  \caption{Oszilloskopbild der Spannung nach dem Schwingkreis(grün) mit der Modulatorspannung(gelb).}
  \label{fig:hnachSchwingkreis}
\end{figure}
\begin{figure}[h]
  \centering
  \includegraphics[width=.9\textwidth]{Oszi_Pics/hnachDiode?.png}
  \caption{Oszilloskopbild der Spannung nach der Diode(grün) mit der Modulatorspannung(gelb).}
  \label{fig:hnachDiode}
\end{figure}
\begin{figure}[h]
  \centering
  \includegraphics[width=.9\textwidth]{Oszi_Pics/hnachTiefpass.png}
  \caption{Oszilloskopbild der Spannung nach dem Tiefpass(grün) mit der Modulatorspannung(gelb).}
  \label{fig:hnachTiefpass}
\end{figure}
