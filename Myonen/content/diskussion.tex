\section{Diskussion}
\label{sec:Diskussion}

Bei der Justage der Verzögerungszeiten ist ein breites Plateau erkennbar, siehe Abbildung \ref{fig:justage}.
Die eingestellte Verzögerungszeit von
\begin{align*}
   T_\text{V} = \SI{3}{\nano\second}
\end{align*}
liegt zentral auf diesem Plateau und kann zuverlässig für die Messung verwendet werden.

Die Messwerte und der lineare Fit der Kalibrationsmessung sind in Abbildung \ref{fig:kalibration} dargestellt.
Die relativen Fehler der Fitparameter lauten
\begin{align}
  \Delta m_\text{rel} \approx \SI{0}{\percent}
\end{align}
für die Steigung und
\begin{align}
  \Delta b_\text{rel} = \SI{8.2}{\percent}
\end{align}
für den Abzissenabschnitt und sind damit sehr gering. Die Umrechnung der Messwerte mittels eines linearen Zusammenhangs ist
damit verlässlich.

Beim Vergleich des theoretisch berechneten Untergrundwerts \eqref{eqn:Utheo} und des numerisch bestimmten
Untergrundwerts \eqref{eqn:Unum} fällt auf, dass diese mit einer Abweichung von
\begin{align}
  \Delta U_\text{rel} = \SI{1.3}{\percent}
\end{align}
nah beieinander liegen, jedoch der Fehler beim numerisch bestimmten Untergrund um zwei Zehnerpotenzen größer ist,
als beim theoretischen. Die numerische Bestimmung ist etwas ungenauer, da das Anpassen von mehr Fitparametern
allgemein ungenauer ist. Bei der theoretischen Berechnung resultiert der Fehler lediglich aus der Poissonabweichung der
Startcounts $N_\text{start}$ und ist daher wesentlich geringer.
Da die verschiedenenen Untergründe dennoch sehr nah beieinander liegen, liegen auch die berechneten Lebensdauern nah beieinander,
die Abweichung ist vernachlässigbar klein. Die Abweichung vom Literaturwert \cite{pdg}
\begin{align}
  \tau = \SI{2.197}{\micro\second}
\end{align}
beträgt jeweils
\begin{align}
  \Delta \tau_\text{rel} = \SI{4.0}{\percent}
\end{align}
und ist damit relativ gering. Ein Grund für die Abweichung nach unten kann unter anderem die Tatsache sein, dass die
Zerfallszeit der Myonen in Materie und nicht im Vakuum gemessen wurde. Das hat zur Folge, dass die negativ geladenen
Myonen eine weitere Option anstatt des Zerfalls haben, sie können nach ihrer Abbremsung von einem Atomkern eingefangen
werden, sodass ein hochangeregtes myonisches Atom entsteht. Dabei entsteht ebenfalls ein Lichtblitz, sodass die Zerfallszeit
des Myons, wenn die zweite Option in Materie vernachlässigt wird, unterschätzt wird.
