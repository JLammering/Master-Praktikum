\section{Auswertung}
\label{sec:Auswertung}

Bei allen Myon-Counts-Messwerten wird ein $\sqrt{N}$-Fehler gemäß der Poissonstatistik
angenommen.

\subsection{Justage}

In den SEVs, den Kabeln und den Diskriminatoren entstehen bei einem Myonsignal unterschiedliche Verzögerungen, sodass es vorkommen kann, dass
am Koinzidenzschalter kein Signal weitergegeben wird, obwohl ein Myon in den Detektor eingetroffen ist. Um dies zu vermeiden werden
vor beiden Diskriminatoren die Verzögerungen justiert. Um die optimale Verzögerung zu bestimmen werden die Verzögerungen variiert und jeweils die Myonen-Counts für ein
$\SI{10}{\second}$-Messintervall aufgenommen. In Abbildung \ref{fig:justage} sind die Messwerte für die Zählrate
gegen die Verzögerungszeitdifferenz aufgetragen. Für das gesamte Experiment wird eine Verzögerungszeitdifferenz von
\begin{align*}
  T_\text{V} = \SI{3}{\nano\second}
\end{align*}
eingestellt, da dieser Wert sehr zentral auf dem Plateau liegt. Von den rot markierten Messpunkten wird das arithmetische
Mittel und die Standardabweichung bestimmt. Es ergibt sich für die Höhe des Plateaus
\begin{align}
  I_\text{Plateau} = \SI{187(12)e-1}{\per\second}.
\end{align}
Außerdem wird jeweils für die schwarz gekennzeichneten Messwerte links vom Plateau und die blau gekennzeichneten Messwerte
rechts vom Plateau eine Ausgleichsgerade bestimmt. Für die Geradengleichung
\begin{align}
  I_i(\Delta t) = m_i \Delta t + b
\end{align}
ergeben sich die Parameter
\begin{align}
  m_\text{li} &= \SI{1.76(19)}{\per\nano\second\per\second} \qquad \qquad &b_\text{li} &= \SI{282(19)e-1}{\per\second}
\end{align}
für den linken Fit und
\begin{align}
  m_\text{re} &= \SI{1.63(18)}{\per\nano\second\per\second} \qquad \qquad &b_\text{re} &= \SI{401(32)e-1}{\per\second}
\end{align}
für den rechten Fit. Um die Halbwertsbreite des Plateaus zu bestimmen werden zunächst jeweils die Schnittpunkte zwischen
halbem Plateaumittelwert und linkem Fit und halbem Plateaumittelwert und rechtem Fit bestimmt. Dabei ergibt sich
\begin{align}
  \Delta t_\text{li} &= \SI{-107(16)e-1}{\nano\second} \qquad \qquad & \Delta t_\text{re} &= \SI{189(29)e-1}{\nano\second}.
\end{align}
Die Halbwertsbreite entspricht etwa der doppelten Breite der Rechtsecksimpulse und lautet
\begin{align}
  \Delta t_{\frac12} = \SI{296(33)}{\nano\second}.
\end{align}

\begin{figure}
  \centering
  \includegraphics{build/justage_koinzidenz.pdf}
  \caption{Messwerte der Justagemessung zur Bestimmung der Halbwertsbreite und der optimalen Verzögerung zwischen den beiden SEVs.
  Die Messwerte werden aufgespalten in Plateau-,Links- und Rechts-Messwerte und es wird jeweils ein linearer Fit durchgeführt.}
  \label{fig:justage}
\end{figure}

\subsection{Kalibration}
\label{sec:kalibration}

Die Myonen-Counts für diverse Zerfallszeitintervalle werden in channels gespeichert.
Daher muss eine Kalibrationsmessung zur Umrechnung von channels in Zerfallszeiten
durchgeführt werden. Die Messwerte sind in Tabelle \ref{tab:kalibration} aufgelistet
und in Abbildung \ref{fig:kalibration} graphisch dargestellt. Es wird eine lineare Ausgleichsrechnung
durchgeführt. Für die Geradengleichung
\begin{align}
  \Delta t (\text{channel}) = m_\text{kal} \cdot \text{channel} + b_\text{kal}
\end{align}
ergeben sich die Parameter
\begin{align}
  m_\text{kal} &= \SI{2.236(1)e-2}{\nano\second} \qquad \qquad &b_\text{kal} &= \SI{4.01(33)e-2}{\nano\second}.
\end{align}
Die resultierende Ausgleichsgerade ist ebenfalls in Abbildung \ref{fig:kalibration} aufgetragen.

\begin{figure}
  \centering
  \includegraphics{build/kalibration.pdf}
  \caption{Messwerte und Ausgleichsgerade der Kalibrationsmessung zur Umrechnung von channels in Zerfallszeiten.}
  \label{fig:kalibration}
\end{figure}

\subsection{Berechnung der theoretischen Untergrungrate für die experimentellen Gegebenheiten}
\label{sec:untergrundtheo}

Bei der Messung der Lebensdauer kosmischer Myonen mit der in \ref{sec:Durchführung} beschriebenen Apparatur
tritt das Problem auf, dass Stopp-signale fälschlicherweise durch ein weiteres eintreffendes Myon ausgelöst werden können,
anstatt durch den Zerfall des eingefallenen Myons. Über die gesamte Messzeit von
\begin{align}
  t_\text{mess} = \SI{85721}{\second}
  \label{eqn:Messzeit}
\end{align}
tritt also ein Untergrund auf, der sich etwa gleichmäßig auf alle Kanäle verteilt, da die Myonen unabhängig voneinander eintreffen.
Die Wahrscheinlichkeit dafür, dass $n$ Myonen bei einem Erwartungswert $\mu(T_\text{S})$ in einer Zeit $T_\text{S}$ in den Detektor einfallen, ist gegeben durch die Poissonverteilung
\begin{align}
  p_\mu(n) = \frac{\mu}{n!} \mathrm{e}^{-\mu}.
\end{align}
Der Erwartungswert $\mu(T_\text{S})$ entspricht dem Wert, wie viele Myonen durchschnittlich in der Suchzeit
\begin{align}
  T_\text{S} = \SI{10.3}{\micro\second}
\end{align}
in den Detektor einfallen. Dieser lässt sich berechnen über
\begin{align}
  \mu(T_\text{S}) = I_\text{mess} \cdot T_\text{S},
  \label{eqn:poisson}
\end{align}
wobei $I_\text{mess}$ die durchschnittliche gemessene Rate, mit der die Myonen eintreffen, ist.
Diese folgt aus der Messzeit \eqref{eqn:Messzeit} und der gesamten gemessenen Anzahl an Startsignalen
\begin{align}
  N_\text{start} = \num{1456661(1207)}
\end{align}
mit
\begin{align}
  I_\text{mess} = \frac{N_\text{start}}{t_\text{mess}} = \SI{16.99(1)}{\per\second}.
\end{align}
Der gesamte Untergrund während der Messzeit $t_\text{mess}$ ergibt sich aus der Wahrscheinlichkeit für
genau ein eintreffendes Myon innerhalb der Suchzeit $T_\text{S}$ gemäß Gleichung \eqref{eqn:poisson} multipliziert mit der
gesamten Anzahl an Startereignissen
\begin{align}
  U_\text{ges} = p_\mu(1) N_\text{start} = \num{254.9(4)}.
\end{align}
Dieser Wert muss noch auf die Anzahl relevanter Kanäle, die innerhalb der Suchzeit $T_\text{S}$ echte Myonzerfallssignale
oder Untergrund messen, normiert werden. Aus der Kalibrationsgeraden in Abschnitt \ref{sec:kalibration} ergibt sich für das
Zeitintervall $[0,T_\text{S}]$ etwa das channel-Intervall $[2,463]$. Der normierte Untergrund lautet daher
\begin{align}
  U_\text{t} = \frac{U_\text{ges}}{463-2} = \num{0.5330(9)}.
  \label{eqn:Utheo}
\end{align}

\subsection{Bestimmung der Lebensdauer kosmischer Myonen}

Die Messwerte der Counts $N$ pro Zeitintervall sind in Abbildung \ref{fig:messung} graphisch dargestellt.
Bei den ersten drei Zeitintervallen bzw. channels werden entgegen der Erwartung einer Exponentialverteilung Null Counts gemessen,
weshalb diese channels vernachlässigt werden. Dieselbe Beobachtung wird bei den channels $>463$ bzw den Zeitintervallen
$>\SI{10.311}{\micro\second}$ gemacht. Diese channels werden ebenfalls abgeschnitten, da die Suchzeit von $T_\text{S}$ dort weder echte Signale
noch Untergrund-Signale zulässt. Der vierte channel wird im Folgenden ebenfalls vernachlässigt, da der gemessene Wert
von $N = \SI{118(11)}{counts}$ sehr stark von den anderen umliegenden Messwerten abweicht, was eher auf einen Messfehler, als auf eine
statistische Schwankung hindeutet.

Mit den übrigen, in Abbildung \ref{fig:messung} in schwarz eingezeichneten Messpunkten wird ein Funktionenfit mit einer modifizierten Exponentialverteilung
\begin{align}
  N_\text{n}(t) = N_{0,\text{n}} \mathrm{e}^{-\lambda_\text{n} t} + U_\text{n}
\end{align}
durchgeführt. Es ergeben sich die Parameter
\begin{align}
  N_{0,\text{n}} &= \SI{40.3(4)}{counts} & \lambda_\text{n} &= \SI{0.4748(3)}{\per\micro\second} \\
  U_\text{n} &= \SI{0.56(8)}{counts}. & &
  \label{eqn:Unum}
\end{align}
Die berechnete Fitfunktion ist ebenfalls in Abbildung \ref{fig:messung} zu sehen.
Aus der Zerfallskonstante ergibt sich die Lebensdauer der kosmischen Myonen
\begin{align}
  \tau_\text{n} = \frac1{\lambda_\text{n}} = \SI{2.11(7)}{\micro\second}.
\end{align}
Bei diesem Vorgehen wurde der Untergrund $U_\text{n}$ numerisch als Fitparameter bestimmt.

\begin{figure}[h]
  \centering
  \includegraphics{build/messung.pdf}
  \caption{Messwerte und Exponentialfit, welcher den Untergrund $U_\text{n}$ als Fitparameter berücksichtigt, der Lebensdauer-Messung.}
  \label{fig:messung}
\end{figure}

Als nächstes wird die Lebensdauer mit dem in Abschnitt \ref{sec:untergrundtheo} theoretisch berechneten Untergrund
\begin{align}
  U_\text{t} = \SI{0.5530(9)}{counts}
\end{align}
bestimmt. Dazu wird dieser Untergrund von jedem nicht vernachlässigtem Messwert abgezogen und erneut ein Exponentialfit,
dieses mal ohne zusätzliche Konstante, gemäß
\begin{align}
  N_\text{t}(t) = N_{0,\text{t}} \mathrm{e}^{-\lambda_\text{t} t}
\end{align}
durchgeführt. Sowohl die um den theoretischen Untergrund korrigierten Messwerte, als auch die neue Fitfunktion
sind in Abbildung \ref{fig:messungtu} graphisch dargestellt. Die zugehörigen berechneten Fitparameter lauten
\begin{align}
  N_{0,\text{t}} &= \SI{40.3(4)}{counts} & \lambda_\text{t} &= \SI{0.4744(1)}{\per\micro\second}.
\end{align}
Damit ergibt sich die Lebensdauer der kosmischen Myonen
\begin{align}
  \tau_\text{t} = \frac1{\lambda_\text{t}} = \SI{2.11(5)}{\micro\second}.
\end{align}
Bei diesem Vorgehen wurde der Untergrund $U_\text{t}$ wie bereits erwähnt theoretisch berechnet.

\begin{figure}[h]
  \centering
  \includegraphics{build/messung_t_u.pdf}
  \caption{Um den Untergrund $U_\text{t}$ verschobene Messwerte und Exponentialfit der Lebensdauer-Messung.}
  \label{fig:messungtu}
\end{figure}

\begin{table}[h]
  \centering
  \begin{tabular}{S S}
    \toprule
    {channel} & {$\Delta t/\si{\micro\second}$}\\
    \midrule
    24 & 0.5 \\
    47 & 1.0 \\
    69 & 1.5 \\
    91 & 2.0 \\
    114 & 2.5 \\
    136 & 3.0 \\
    158 & 3.5 \\
    181 & 4.0 \\
    203 & 4.5 \\
    225 & 5.0 \\
    248 & 5.5 \\
    270 & 6.0 \\
    292 & 6.5 \\
    315 & 7.0 \\
    337 & 7.5 \\
    360 & 8.0 \\
    382 & 8.5 \\
    404 & 9.0 \\
    427 & 9.5 \\
    445 & 9.9 \\
    \bottomrule
  \end{tabular}
  \caption{Messwerte der Kalibrationsmessung.}
  \label{tab:kalibration}
\end{table}
