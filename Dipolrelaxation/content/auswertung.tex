\section{Auswertung}
\label{sec:Auswertung}

\subsection{Bestimmung der Heizraten}

Die Messwerte der beiden durchgeführten Strom-Temperaturmessungen an der mit
Strontium dotierten KBr-Probe sind im Anhang in den Tabellen \ref{tab:mess1}
und \ref{tab:mess2} zu finden. Alle Messwerte liegen dabei zeitlich eine Minute
auseinander. In den Abbildungen \ref{fig:heizrate2} und \ref{fig:heizrate15} sind
die Temperaturmesswerte gegen die Zeit für die beiden angestrebten Heizraten
von $\SI{2}{\celsius\per\minute}$ und $\SI{1.5}{\celsius\per\minute}$ aufgetragen.
Es wird jeweils ein linearer Fit durchgeführt, wobei sich die Abzissenabschnitte
\begin{align}
  n_{2} &= \SI{-46.4(1)}{\celsius}\\
  n_{1.5} &= \SI{-49.7(1)}{\celsius}
\end{align}
und die Steigungen
\begin{align}
  m_{2} &= \SI{0.03233(5)}{\celsius\per\second}\\
  m_{1.5} &= \SI{0.02411(3)}{\celsius\per\second}
\end{align}
ergeben. Aus den Steigungen folgen sofort die Heizraten
\begin{align}
  b_{2} &= \SI{0.03233(5)}{\celsius\per\second} \approx \SI{2}{\celsius\per\minute} \\
  b_{1.5} &= \SI{0.02411(3)}{\celsius\per\second} \approx \SI{1.5}{\celsius\per\minute}.
  \label{eqn:heizrate}
\end{align}

\begin{figure}
  \centering
  \includegraphics{build/heizrate_2schritt.pdf}
  \caption{Temperaturmesswerte aufgetragen gegen die Zeit für die angestrebte Heizrate von $\SI{2}{\celsius\per\minute}$. Die Steigung
  des Fits entspricht der Heizrate $b_{2}$.}
  \label{fig:heizrate2}
\end{figure}

\begin{figure}
  \centering
  \includegraphics{build/heizrate_1_5schritt.pdf}
  \caption{Temperaturmesswerte aufgetragen gegen die Zeit für die angestrebte Heizrate von $\SI{1.5}{\celsius\per\minute}$. Die Steigung
  des Fits entspricht der Heizrate $b_{1.5}$.}
  \label{fig:heizrate15}
\end{figure}

\subsection{Offsetbereinigung der Messwerte}

Die Stromtemperaturmesswerte sind in den Abbildungen \ref{fig:messwerte2} und
\ref{fig:messwerte15} graphisch für beide Heizraten aufgetragen. Es wird
aufgrund des Verlaufs der Messwerte und der erwarteten Theoriekurve ein
exponentieller Offset angenommen. Um die Messwerte um diesen Offset zu bereinigen
werden jeweils Messwerte vor dem ersten Peak und zwischen den beiden erkennbaren
Peaks ausgewählt und exponentiell gefittet. Die selektierten Messwerte sind in beiden
Graphen jeweils rot dargestellt. Beim Fit mit der Exponentialfunktion
\begin{align}
  I_\text{offset}(T)/\si{\pico\ampere} = \mathrm{e}^{\alpha(T-\mu)}
\end{align}
ergeben sich die Parameter
\begin{align}
  \mu_{2} &= \SI{-46(1)}{\celsius} \\
  \mu_{1.5} &= \SI{-45(2)}{\celsius}
\end{align}
und
\begin{align}
  \alpha_{2} &= \SI{0.0382(8)}{\per\celsius} \\
  \alpha_{1.5} &= \SI{0.034(2)}{\per\celsius}.
\end{align}
In den Abbildungen \ref{fig:messwerte2off} und \ref{fig:messwerte15off} sind
jeweils die bereinigten Messwerte für den ersten Peak zu sehen. Der zweite Peak ist
für die Auswertung uninteressant, da er nicht aus der in diesem Experiment
betrachteten Dipolrelaxation der Strontium-Leerstellen-Dipole resultiert.
Die bei den letzteren Graphen rot dargestellten Messwerte beschreiben jeweils den
Stromstärkeanstieg des Relaxationspeaks und werden für die erste Bestimmung der
Aktivierungsenergie $W$ hinzugezogen. Die eingezeichneten Parabelfits sind relevant
für die Bestimmung der charakteristischen Relaxationszeit.

\begin{figure}
  \centering
  \includegraphics{build/stromtemp_2.pdf}
  \caption{Strommesswerte aufgetragen gegen die Temperatur für die Heizrate $b_{2}$. Es ist ein exponentieller
  Untergrund an die Messwerte, die ohne Offset vermeindlich bei Null liegen, gefittet.}
  \label{fig:messwerte2}
\end{figure}

\begin{figure}
  \centering
  \includegraphics{build/stromtemp_1_5.pdf}
  \caption{Strommesswerte aufgetragen gegen die Temperatur für die Heizrate $b_{1.5}$. Es ist ein exponentieller
  Untergrund an die Messwerte, die ohne Offset vermeindlich bei Null liegen, gefittet.}
  \label{fig:messwerte15}
\end{figure}

\begin{figure}
  \centering
  \includegraphics{build/stromtemp_2_off.pdf}
  \caption{Offsetbereinigte Messwerte des ersten Peaks für die Heizrate $b_{2}$. Die Messwerte, die zur Berechnung der Aktivierungsenergie mit der
  ersten Methode benötigt werden, sind in rot dargestellt. Die hellblauen Messwerte werden durch eine Parabel angenähert um
  die Stelle des Peakmaximums zu ermitteln.}
  \label{fig:messwerte2off}
\end{figure}

\begin{figure}
  \centering
  \includegraphics{build/stromtemp_1_5_off.pdf}
  \caption{Offsetbereinigte Messwerte des ersten Peaks für die Heizrate $b_{1.5}$. Die Messwerte, die zur Berechnung der Aktivierungsenergie mit der
  ersten Methode benötigt werden, sind in rot dargestellt. Die hellblauen Messwerte werden durch eine Parabel angenähert um
  die Stelle des Peakmaximums zu ermitteln.}
  \label{fig:messwerte15off}
\end{figure}

\subsection{Bestimmung der Aktivierungsenergie aus dem Anstieg des Peaks}

Die in den Abbildungen \ref{fig:messwerte2off} und \ref{fig:messwerte15off} rot
eingezeichneten Messwerte sind in den Abbildungen \ref{fig:messwerte2offW1} und
\ref{fig:messwerte15offW1} logarithmisch gegen die reziproke Temperatur
aufgetragen. Nach der theoretisch hergeleiteten Gleichung \eqref{eqn:W1formel}
ergibt sich für die entsprechend aufgetragenen Messwerte ein linearer Verlauf.
Im Folgenden werden Temperaturen in $\si{\kelvin}$ angegeben, da die reziproke
Temperatur nur so sinnvoll definiert ist. Es wird eine lineare Regression durchgeführt,
bei welcher sich die Parameter
\begin{align}
  n_{2} &= \num{54(2)} \\
  n_{1.5} &= \num{72(6)}
\end{align}
für die Abzissenabschnitte und
\begin{align}
  m_{2} &= \SI{-1.32(5)e04}{\per\kelvin} \\
  m_{1.5} &= \SI{-1.8(1)e04}{\per\kelvin}
\end{align}
für die Steigungen ergeben. Aus den Steigungen ergeben sich die Aktivierungsenergien
\begin{align}
  W_{2} &= m_{2} \, \cdot k_\text{B} = \SI{1.82(7)e-19}{\joule} = \SI{1.13(4)}{\electronvolt} \\
  W_{1.5} &= m_{1.5} \cdot k_\text{B} = \SI{2.4(2)e-19}{\joule} = \SI{1.5(1)}{\electronvolt}.
\end{align}
Wird nun noch das arithmetische Mittel gebildet, so ergibt sich die mit der ersten
Auswertungsmethode bestimmte Aktivierungsenergie
\begin{align}
  W_{\text{I}} &= \SI{2.1(6)e-19}{\joule} = \SI{1.3(4)}{\electronvolt}.
\end{align}

\begin{figure}
  \centering
  \includegraphics{build/stromtemp_2_off_W1.pdf}
  \caption{Bestimmung der Aktivierungsenergie aus dem Anstieg des ersten Peaks für die Heizrate $b_{2}$.
  Die Messwerte werden logarithmiert, gegen die reziproke Temperatur aufgetragen und schließlich linear gefittet.}
  \label{fig:messwerte2offW1}
\end{figure}

\begin{figure}
  \centering
  \includegraphics{build/stromtemp_1_5_off_W1.pdf}
  \caption{Bestimmung der Aktivierungsenergie aus dem Anstieg des ersten Peaks für die Heizrate $b_{1.5}$.
  Die Messwerte werden logarithmiert, gegen die reziproke Temperatur aufgetragen und schließlich linear gefittet.}
  \label{fig:messwerte15offW1}
\end{figure}

\subsection{Bestimmung der Aktivierungsenergie aus dem gesamten Peak}

Um die Aktivierungsenergie mit Hilfe des gesamten Peaks, also letzlich mit Hilfe
von Gleichung \ref{eqn:W2formel} zu ermitteln, muss zunächst eine numerische Methode
gefunden werden, mit welcher das auftretende Integral mit Hilfe der Messwerte
approximiert werden kann. Dazu werden aufeinander folgende Messwerte linear
verbunden und die Fläche der eingeschlossenen Trapeze gemäß
\begin{align}
  \int_{T_1}^{T_N} I(T) \mathrm{d}T \approx \sum_{i=1}^{N-1} \frac12 \frac{I_{i+1}+I_{i}}{T_{i+1}-T{i}},
\end{align}
wobei die Summe über alle Messpunkte innerhalb der Integrationsgrenzen läuft, berechnet.
Die Cutoff-Temperaturen, an welchen die Stromstärke nach dem Peak näherungsweise wieder auf Null
abgefallen sein sollte, werden auf
\begin{align}
  T^*_{2} &= \SI{1.7}{\celsius} \\
  T^*_{1.5} &= \SI{1.0}{\celsius}
\end{align}
gesetzt und entsprechen den letzten noch eingezeichneten Messpunkten in den Abbildungen
\ref{fig:messwerte2off} und \ref{fig:messwerte15off}. Für die folgenden Auswertungsschritte
werden bei der ersten Heizrate $b \approx \SI{2}{\celsius\per\minute}$ die ersten beiden
und bei der zweiten Heizrate $b \approx \SI{1.5}{\celsius\per\minute}$ die ersten neun Messwerte
nicht berücksichtigt, da die reziproke Stromstärke und der Logarithmus auf der rechten Seite
von Gleichung \eqref{eqn:W2formel} divergieren bzw. nicht definiert sind.
In den Abbildungen \ref{fig:messwerte2offW2} und \ref{fig:messwerte15offW2} sind die Ergebnisse
\begin{align}
  f(I) = \ln \left(\frac{\int_{T_i}^{T^*} I(T)\dif T}{I_i}\right)
  \label{eqn:auftrag}
\end{align}
der übrigen Messwerte $I_{i}$ gegen die reziproke Temperatur aufgetragen. Es wird erneut
jeweils für beide Heizraten ein linearer Fit durchgeführt. Dabei ergibt sich
\begin{align}
  n_{2} &= \num{-48(2)} \\
  n_{1.5} &= \num{-42(1)}
\end{align}
für die Abzissenabschnitte und
\begin{align}
  m_{2} &= \SI{1.29(4)e04}{\per\kelvin} \\
  m_{1.5} &= \SI{1.14(3)e04}{\per\kelvin}
\end{align}
für die Steigungen. Daraus ergeben sich die Aktivierungsenergien
\begin{align}
  W_{2} &= m_{2} \, \cdot k_\text{B} = \SI{1.78(6)e-19}{\joule} = \SI{1.11(4)}{\electronvolt} \\
  W_{1.5} &= m_{1.5} \cdot k_\text{B} = \SI{1.57(5)e-19}{\joule} = \SI{0.98(3)}{\electronvolt}.
\end{align}
Mitteln führt auf die mit der zweiten Auswertungsmethode berechnete Aktivierungsenergie
\begin{align}
  W_{\text{II}} = &= \SI{1.7(2)e-19}{\joule} = \SI{1.1(1)}{\electronvolt}.
\label{eqn:akt2}
\end{align}

\begin{figure}
  \centering
  \includegraphics{build/stromtemp_2_off_W2.pdf}
  \caption{Ermittlung der Aktivierungsenergie aus dem gesamten ersten Peak für die Heizrate $b_{2}$. Die Messwerte
  werden mit Relation \eqref{eqn:auftrag} umgeformt, gegen die reziproke Temperatur aufgetragen und anschließend
  linear gefittet.}
  \label{fig:messwerte2offW2}
\end{figure}

\begin{figure}
  \centering
  \includegraphics{build/stromtemp_1_5_off_W2.pdf}
  \caption{Ermittlung der Aktivierungsenergie aus dem gesamten ersten Peak für die Heizrate $b_{1.5}$. Die Messwerte
  werden mit Relation \eqref{eqn:auftrag} umgeformt, gegen die reziproke Temperatur aufgetragen und anschließend
  linear gefittet.}
  \label{fig:messwerte15offW2}
\end{figure}

\subsection{Bestimmung der charakteristischen Relaxationszeit}

Um die charakteristische Relaxationszeit $\tau_0$ aus den Messdaten zu bestimmen muss
zunächst ein Zusammenhang zur kritischen Temperatur, an welcher der Strom-Peak
sein Maximum erreicht, hergeleitet werden: Differenzieren der Stromstärke aus
Gleichung \eqref{eqn:allgjt} führt auf
\begin{align}
  I'(T) &= I(T) \cdot \left(\frac{W}{k_\text{B} T^2} - \frac1{b \tau_0} \frac{\mathrm{d}}{\mathrm{d}T} \int_{T_0}^{T} \mathrm{e}^{-\frac{W}{k_\text{B} T'}} \mathrm{d}{T'}\right) \\
  &= I(T) \cdot \left(\frac{W}{k_\text{B} T^2} - \frac1{b \tau_0} \frac{\mathrm{d}}{\mathrm{d}T} \left(F(T) - F(T_0)\right)\right) \\
  &= I(T) \cdot \left(\frac{W}{k_\text{B} T^2} - \frac1{b \tau_0} \mathrm{e}^{-\frac{W}{k_\text{B} T}}\right)
\end{align}
und nach Gleichsetzen mit Null auf die Relation
\begin{align}
  \tau_0 = \frac{k_\text{B} T_\text{max}^2}{W b} \mathrm{e}^{-\frac{W}{k_\text{B} T_\text{max}}}.
  \label{eqn:tau0}
\end{align}
Um die kritische Temperatur $T_\text{max}$ zu ermitteln werden Parabelfits mit den offsetbereinigten
Messwerten des jeweils ersten Peaks durchgeführt. In den Abbildungen \ref{fig:messwerte2off} und \ref{fig:messwerte15off}
sind die für den Fit genutzten Messwerte in hellblau und die berechneten Fitfunktionen in dunkelblau dargestellt.
Zur Funktionsvorschrift
\begin{align}
  P(T) = a T^2 + b T + c
\end{align}
ergeben sich die Fitparameter
\begin{align*}
  a_{2} &= \SI{-0.177(8)}{\pico\ampere\per\celsius\squared} & b_{2} &= \SI{-2.8(2)}{\pico\ampere\per\celsius} & c_{2} &= \SI{-3(1)}{\pico\ampere} \\
  a_{1.5} &= \SI{-0.099(7)}{\pico\ampere\per\celsius\squared} & b_{1.5} &= \SI{-2.9(2)}{\pico\ampere\per\celsius} & c_{1.5} &= \SI{-11(1)}{\pico\ampere}
\end{align*}
und daraus die kritischen Temperaturen
\begin{align}
  T_{\text{max},2} &= -\frac{b_{2}}{2 a_{2}} = \SI{-12(1)}{\celsius} = \SI{285(1)}{\kelvin}\\
  T_{\text{max},15} &= -\frac{b_{1.5}}{2 a_{1.5}} = \SI{-15(2)}{\celsius} = \SI{288(2)}{\kelvin}.
\end{align}
Werden nun die kritischen Temperaturen, die jeweiligen Heizraten aus Gleichung \eqref{eqn:heizrate} und die
Aktivierungsenergie aus Gleichung \eqref{eqn:akt2} in die rechte Seite von Formel \eqref{eqn:tau0}
eingesetzt, so ergeben sich die charakteristischen Relaxationszeiten
\begin{align}
  \tau_{0,(2)} &= \SI{1(3)e-3}{\femto\second} \\
  \tau_{0,(1.5)} &= \SI{1(3)e-3}{\femto\second}.
\end{align}
Diese entsprechen ihrem arithmetischen Mittel
\begin{align}
  \tau_0 &= \SI{1(3)e-3}{\femto\second}.
  \label{eqn:tau0}
\end{align}
Abschließend zur Auswertung ist in Abbildung \ref{fig:relaxationszeit} noch einmal die Relaxationszeit
\begin{align}
  \tau(T) = \tau_0 \cdot \mathrm{e}^{\frac{W}{k_\text{B} T}}
\end{align}
mit durch Gleichung \eqref{eqn:tau0} gegebenem $\tau_0$ und durch Gleichung \eqref{eqn:akt2} gegebenem
$W$ graphisch dargestellt.

\begin{figure}
  \centering
  \includegraphics{build/relax.pdf}
  \caption{Relaxationszeit in Abhängigkeit von der Temperatur für die ermittelten Konstanten $\tau_0$ aus
  Gleichung \eqref{eqn:tau0} und $W$ aus Gleichung \eqref{eqn:akt2}.}
  \label{fig:relaxationszeit}
\end{figure}
