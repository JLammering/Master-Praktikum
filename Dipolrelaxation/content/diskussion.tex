\section{Diskussion}
\label{sec:Diskussion}

Die Heizraten können mit einem relativen Fehler von
\begin{align}
  \Delta_{b,2} &= \SI{0.2}{\percent}, \\
  \Delta_{b,2} &= \SI{0.1}{\percent}
\end{align}
sehr genau angegeben werden. Allerdings sind in den Ergebnissen trotzdem statistische Fehler,
die aus der Schwankung der Heizraten resultieren, zu erwarten. Das liegt unter anderem
daran, dass die Stromstärke des Heizstroms regelmäßig korrigiert werden musste und sich erst
nach einiger Zeit herausstellte, ob die eingestellte Stromstärke optimal für die angestrebte
Heizrate war. Das bereinigen der Messwerte um einen exponentiellen Untergrund hat ebenfalls gut funktioniert,
was an den geringen Fehlern der Anpassungsparameter aus den Gleichungen \eqref{eqn:disk1} und \eqref{eqn:disk2} deutlich
wird. Der Grund für den mit der Temperatur ansteigenden Untergrundstrom ist möglicherweise, dass
Elektronen aus der erhitzten Probe und umliegendem Kondensator thermisch angeregt werden und als
Untergrundstrom messbar sind. Werden die Messwerte des zweiten Maximums betrachtet, so fällt auf, dass der
angenähert Untergrund dort über den Werten liegt. Dies ist allerdings irrelevant, da diese Messwerte ohnehin
nicht einbezogen werden.

Die Ermittlung der Aktivierungsenergie aus dem Maximumsanstieg ist wesentlich
unzuverlässiger als die Ermittlung der Aktivierungsenergie aus dem gesamten Maximum. Das ist vor Allem an
den relativen Fehlern
\begin{align}
  \Delta_{W,\text{I}} &= \SI{30.8}{\percent}, \\
  \Delta_{W,\text{II}} &= \SI{9.1}{\percent}
\end{align}
erkennbar. Die Abweichung vom Literaturwert \cite{paper}
\begin{align}
  W_\text{lit} = \SI{0.66}{\electronvolt}
\end{align}
ist mit
\begin{align}
  \Delta_{W,\text{I} - \text{lit}} &= \SI{97}{\percent} \\
  \Delta_{W,\text{II} - \text{lit}} &= \SI{83}{\percent} \\
\end{align}
ist in beiden Fällen relativ groß.
Der Grund dafür, dass die zweite Methode ein besseres Ergebnis liefert, ist einerseits, dass wesentlich
mehr Messwerte einbezogen werden können und andererseits, dass die zugehörige Theoriegleichung nahezu exakt ist.
Bei der Auswertung mit der ersten Methode müssen die Messwerte sorgfältig ausgewählt werden, sodass
die Näherungsformel \eqref{eqn:W1formel} gültig ist. Ein Grund für die Literaturwertabweichung ist möglicherweise,
dass die Annahme eines exponentiellen Untergrunds problematisch ist.

Die Bestimmung der Relaxationszeit stellt sich als sehr ungenau heraus, wenn man die relative Abweichung
\begin{align}
  \Delta_{\tau} = \SI{300}{\percent}
\end{align}
betrachtet. Außerdem weicht der berechnete Wert weicht um vier Zehnerpotenzen vom Literaturwert \cite{paper}
\begin{align}
  \tau_\text{lit} = \SI{4 e-14}{\second}
\end{align}
ab. Das liegt weniger an Messungenauigkeiten bei der Parabelanpassung, sondern eher an der Tatsache, dass
die Aktivierungsenergie exponentiell eingeht und der Fehler sich entsprechend vergrößert. Hinzu kommt, dass
der Wert der charakteristischen Relaxationszeit im Bereich $\SI{e-14}{\second}$ liegt und entsprechend
schwierig genau zu ermitteln ist.
