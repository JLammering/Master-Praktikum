\section{Auswertung}
\label{sec:Auswertung}

\subsection{Thermisches Rauschen eines Widerstands}

\paragraph{Einfache Schaltung}

Benutzt wird hier die Schaltung aus Abbildung \ref{fig:???}, die hier als einfache Schaltung bezeichnet wird. Um aus dem thermischen Rauschen eines Widerstands die Boltzmann Konstante $k$ zu bestimmen, muss zunächst eine Eichmesssung durchgeführt werden. Mit den Werten aus Tabelle \ref{tab:durchlasseinfach} ergibt sich normiert auf die Amplitude und die Nachverstärkung $V_\text{N}$ die Durchlasskurve aus Abbildung \ref{fig:plotdurchlasseinfach}.
\begin{figure}
  \centering
  \includegraphics{build/plotdurchlasseinfach.pdf}
  \caption{Durchlasskurve der Apparatur nach einfacher Schaltung aus Abbildung \ref{fig:???}.}
  \label{fig:plotdurchlasseinfach}
\end{figure}
Es werden folgende Einstellungsparameter gewählt:
\begin{align*}
  U_\text{Amplitude} = \SI{200}{\milli\volt} \quad \Delta \nu_\text{Bandfilter} = \num{1}-\SI{50}{\kilo\hertz}\quad V_= = 10 \quad V_\text{V} = 1000 \quad V_\text{N} = 1.
\end{align*}
Außerdem wird ein Abschwächer mit $V_\text{Abschwächung} = 1/1000$ vorgeschaltet. Die Vorverstärkung und die Gleichspannungsverstärkung werden nicht rausgerechnet, da sie in der folgenden Messung gleich bleiben. Das Integral der Kurve aus Abbildung \ref{fig:plotdurchlasseinfach} ergibt, berechnet mit der Simpsonregel, die Apparatekonstante:
\begin{align}
  A = \SI{495496239571}{\hertz}.
\end{align}

Damit kann nun die Boltzmann Konstante berechnet werden. Mit den Widerständen $R_1$ und $R_2$ wurden die Werte aus den Tabellen \ref{tab:widerstand1} und \ref{tab:widerstand2} aufgenommen. Die Vorverstärkung, Gleichspannungsverstärkung und der Filterbereich des Bandfilters werden gleichgelassen. Die Temperatur ist $T = \SI{23(2)}{\celsius}$.
Aus den Werten von $\overline{U_\text{a}^2}$ normiert auf die Nachverstärkung ergeben sich die Graphen in den Abbildungen \ref{fig:plotWiderstand1} und \ref{fig:plotWiderstand2}.
\begin{figure}
  \centering
  \includegraphics[height=7cm]{build/plotWiderstand1.pdf}
  \caption{Plot der Ausgangsspannung normiert auf die Nachverstärkung in Abhängigkeit vom Wert des Widerstands $R_1$.}
  \label{fig:plotWiderstand1}
\end{figure}
\begin{figure}
  \centering
  \includegraphics[height=7cm]{build/plotWiderstand2.pdf}
  \caption{Plot der Ausgangsspannung normiert auf die Nachverstärkung in Abhängigkeit vom Wert des Widerstands $R_2$.}
  \label{fig:plotWiderstand2}
\end{figure}
Ein linearer Fit nach
\begin{align}
  \overline{U_\text{a}^2} &= m R\\
\intertext{ergibt für $R_1$:}
  m_1 &= \SI{3.0(1)e-6}{\hertz\joule}
\intertext{und für $R_2$:}
  m_2 &= \SI{6.82(6)e-9}{\hertz\joule}
\end{align}
Aus Formel \eqref{eqn:???} ergibt sich für die Boltzmann Konstante:
\begin{align*}
\intertext{für $R_1$:}
  k_{1,\text{B}} &= \frac{m_1}{4 A T} = \SI{5.1(3)e-21}{\joule\per\kelvin}
\intertext{mit der Abweichung vom Theoriewert:}
\Delta_{k_{1,\text{B}}} &= \SI{3.7(2)e4}{\percent}
\intertext{und für $R_2$:}
  k_{2,\text{B}} &= \frac{m_2}{4 A T} = \SI{1.16(1)e-23}{\joule\per\kelvin}.\\
  \Delta_{k_{2,\text{B}}} &= \SI{15(1)}{\percent}
\end{align*}


Zuletzt wird die Schaltung noch mit einem $\SI{0}{\ohm}$ Widerstand kurzgeschlossen. Dabei ergeben sich die Messwerte aus Tabelle \ref{tab:kurzschlusseinfach}.
\begin{table}
  \centering
  \begin{tabular}{S S}
    \toprule
    {$V_N$} & {$\overline{U_\text{a}^2}\:/\:\si{\milli\volt\squared}$}\\
    \midrule
    1 & -4\\
    2 & -4\\
    5 & -4\\
    10 & -4\\
    20 & -4\\
    50 & -4\\
    100 & -4\\
    200 & 9\\
    500 & 80\\
    1000 & 315\\
    \bottomrule
  \end{tabular}
  \caption{Amplituden der einfachen kurzgeschlossenen Schaltung.}
  \label{tab:kurzschlusseinfach}
\end{table}


\paragraph{Korrelatorschaltung}

Jetzt wird eine Schaltung nach Abbildung \ref{fig:???} aufgebaut; im Folgenden Korrelatorschaltung genannt. Zur Eichung wird erneut eine Durchlasskurve aufgenommen. Die eingestellten Parameter sind:
\begin{align*}
  U_\text{Amplitude} = \SI{500}{\milli\volt} \quad \nu_\text{m} = \SI{5}{\kilo\hertz}\quad V_= = 10 \quad V_\text{V} = 1000.
\end{align*}
Den beiden Eingängen der Vorverstärker wird jeweils ein Abschwächer mit $V_\text{Abschwächung} = 1/1000$ vorgeschaltet.
Die Messwerte für die Eichmesssung sind in Tabelle \ref{tab:durchlasskorr} eingetragen.
Diese Werte, normiert auf die Amplitude und die Nachverstärkung, sind als Durchlasskurve in Abbildung \ref{fig:plotdurchlasskorr} gezeigt.
\begin{figure}
  \centering
  \includegraphics{build/plotdurchlassKorrelator.pdf}
  \caption{Durchlasskurve der Apparatur nach der Korrelatorschaltung aus Abbildung \ref{fig:???}.}
  \label{fig:plotdurchlasskorr}
\end{figure}
Die Vorverstärkung und die Gleichspannungsverstärkung werden nicht rausgerechnet, da sie in der folgenden Messung gleich bleiben. Das Integral der Kurve aus Abbildung \ref{fig:plotdurchlasskorr} ergibt, berechnet mit der Simpsonregel, die Apparatekonstante:
\begin{align}
  A = \SI{7389018051}{\hertz}.
\end{align}

Damit kann nun die Boltzmann Konstante berechnet werden. Mit den Widerständen $R_1$ und $R_2$ wurden die Werte aus den Tabellen \ref{tab:widerstand1korr} und \ref{tab:widerstand2korr} aufgenommen. Die Vorverstärkung, Gleichspannungsverstärkung und die Mittenfrequenz der Selektionsverstärker werden gleichgelassen. Die Temperatur ist $T = \SI{20(2)}{\celsius}$.
Aus den Werten von $\overline{U_\text{a}^2}$, normiert auf die Nachverstärkung sowie die hier eingestellte Verstärkung des Selektionsverstärkers $V_\text{S} = 10$, ergeben sich die Graphen in den Abbildungen \ref{fig:plotWiderstand1korr} und \ref{fig:plotWiderstand2korr}.
\begin{figure}
  \centering
  \includegraphics[height=7cm]{build/plotWiderstand1Korr.pdf}
  \caption{Plot der Ausgangsspannung normiert auf die Nachverstärkung in Abhängigkeit vom Wert des Widerstands $R_1$.}
  \label{fig:plotWiderstand1korr}
\end{figure}
\begin{figure}
  \centering
  \includegraphics[height=7cm]{build/plotWiderstand2Korr.pdf}
  \caption{Plot der Ausgangsspannung normiert auf die Nachverstärkung in Abhängigkeit vom Wert des Widerstands $R_2$.}
  \label{fig:plotWiderstand2korr}
\end{figure}
Ein linearer Fit nach
\begin{align}
  \overline{U_\text{a}^2} &= m R\\
\intertext{ergibt für $R_1$:}
  m_1 &= \SI{1.10(1)e-10}{\hertz\joule}
\intertext{und für $R_2$:}
  m_2 &= \SI{1.113(5)e-10}{\hertz\joule}
\end{align}
Aus Formel \eqref{eqn:???} ergibt sich für die Boltzmann Konstante:
\begin{align*}
\intertext{für $R_1$:}
  k_{1,\text{B}} &= \frac{m_1}{4 A T} = \SI{1.26(2)e-23}{\joule\per\kelvin}\\
  \Delta_{k_{1,\text{B}}} &= \SI{9(1)}{\percent}
\intertext{und für $R_2$:}
  k_{2,\text{B}} &= \frac{m_2}{4 A T} = \SI{1.27(1)e-23}{\joule\per\kelvin}.\\
  \Delta_{k_{2,\text{B}}} &= \SI{8(1)}{\percent}
\end{align*}

Zuletzt wird die Schaltung noch mit einem $\SI{0}{\ohm}$ Widerstand kurzgeschlossen. Dabei ergeben sich die Messwerte aus Tabelle \ref{tab:kurzschlusskorr}.
\begin{table}
  \centering
  \begin{tabular}{S S}
    \toprule
    {$V_N$} & {$\overline{U_\text{a}^2}\:/\:\si{\milli\volt\squared}$}\\
    \midrule
    50 & -5\\
    100 & -9\\
    200 & -26\\
    500 & -140\\
    1000 & -390\\
    \bottomrule
  \end{tabular}
  \caption{Amplituden der einfachen Korrelatorschaltung.}
  \label{tab:kurzschlusskorr}
\end{table}

\paragraph{Rauschzahl}

Jetzt kann für die Korrelatorschaltung noch die Rauschzahl bei $R = \SI{500}{\ohm}$ bestimmt werden. Da $\nu_\text{m} = \SI{5}{\kilo\hertz}$ gewählt wird, ist $\Delta \nu = \SI{700.7}{\hertz}$. Die Temperatur ist erneut $T = \SI{20(2)}{\celsius}$. Die Verstärkungsfaktoren sind:
\begin{align*}
  V_\text{N} = 200 \quad V_\text{S} = 10 \quad V_= = 10 \quad V_\text{V} = 1000.
\end{align*}
Damit ergibt sich mit dem Messwert aus Tabelle \ref{tab:widerstand1korr} und der Formel \eqref{eqn:???} die Rauschzahl
\begin{align}
  F = \num{0.98(5)}.
\end{align}

\subsection{Untersuchung einer Hochvakuumdiode mit Reinmetallkathode}

\paragraph{Kennlinien}
\begin{figure}
  \centering
  \includegraphics[height=7cm]{build/plotKennlinie1.pdf}
  \caption{Die Kennlinie der Hochvakuumdiode mit Reinmetallkathode bei einem Heizstrom von $I_\text{heiz} = \SI{0.9}{\ampere}$.}
  \label{fig:plotKennlinie1}
\end{figure}
\begin{figure}
  \centering
  \includegraphics[height=7cm]{build/plotKennlinie3.pdf}
  \caption{Die Kennlinie der Hochvakuumdiode mit Reinmetallkathode bei einem Heizstrom von $I_\text{heiz} = \SI{0.95}{\ampere}$.}
  \label{fig:plotKennlinie3}
\end{figure}
\begin{figure}
  \centering
  \includegraphics[height=7cm]{build/plotKennlinie2.pdf}
  \caption{Die Kennlinie der Hochvakuumdiode mit Reinmetallkathode bei einem Heizstrom von $I_\text{heiz} = \SI{1}{\ampere}$.}
  \label{fig:plotKennlinie2}
\end{figure}
In den Abbildungen \ref{fig:plotKennlinie1}, \ref{fig:plotKennlinie3} und \ref{fig:plotKennlinie2} sind die Kennlinien der Diode mit Reinmetallkathode bei verschiedenen Heizströmen gezeigt. So kann abgeschätzt werden, ab welcher Anodenspannung sich die Diode im Sättigungsbereich befindet.
Die Werte zu den Abbildungen finden sich in den Tabellen \ref{tab:kennlinie1}, \ref{tab:kennlinie2} und \ref{tab:kennlinie3}.
In den folgenden Untersuchungen an dieser Diode wird der Heizstrom $I_\text{heiz} = \SI{0.9}{\ampere}$ bei einer Anodenspannung von $U_\text{anode} = \SI{140}{\volt}$
und einem Anodenstrom von $I_\text{anode} = \SI{0.5}{\milli\ampere}$ gewählt.

\paragraph{Bestimmung der Elementarladung}

Zur Bestimmung der Elementarladung wird bei der Diode der Anodenstrom variiert und dann die Rauschspannung abgelesen. Es galten folgende Parameter:
\begin{align*}
  \nu_\text{m} = \SI{340}{\kilo\hertz} \quad R_\text{i} = \SI{4680(1)}{\ohm} \quad V_= = 10 \quad V_\text{V} = 1000\\
  U_\text{anode} = \SI{140}{\volt} \quad I_\text{heiz} = \SI{0.9}{\ampere}.
\end{align*}
Die aufgenommenen Werte sind in Tabelle \ref{tab:elementar} zu finden. Nachdem alle Verstärkungen rausgerechnet sind und $U_\text{a}^2$ nach
\begin{align}
  I_\text{a}^2 = \frac{U_\text{a}^2}{R_\text{i}^2}
\end{align}
in den Rauschstrom umgerechnet ist, werden die Werte in Abbildung \ref{fig:plotElement} gezeigt.
\begin{figure}
  \centering
  \includegraphics[height=7cm]{build/plotElement.pdf}
  \caption{Plot der Werte für die Bestimmung der Elementarladung.}
  \label{fig:plotElement}
\end{figure}
Hier wird nach
\begin{align}
  I_\text{a}^2 = m I_0
\end{align}
eine lineare Ausgleichsrechnung durchgeführt.
Dies ergibt:
\begin{align*}
  m &= \SI{3.54(2)e-15}{\coulomb\hertz}
\intertext{im Vergleich mit \eqref{eqn:???} ergibt sich:}
  e_0 &= \SI{7.3(1)e-20}{\coulomb}.
\intertext{Die Abweichung zum Theoriewert beträgt:}
  \Delta_{e_0} &= \SI{54.7(8)}{\percent}.
\end{align*}

\paragraph{Rauschspektrum}

Die Werte für das Rauschspektrum der Reinmetallkathode sind in Tabelle \ref{tab:rauschspektrumrein} zu sehen.
Außerdem sind
% für den Bandfilter, also die Frequenzen \num{460} bis \SI{120}{\kilo\hertz},
die Parameter so gewählt:
\begin{align*}
  V_\text{V} = 1000 \quad V_= = 10 \quad I_0 = \SI{0.9}{\milli\ampere} \quad R = \SI{2200}{\ohm} \quad I_\text{heiz} = \SI{0.9}{\ampere}\quad U_\text{anode} = \SI{140}{\volt}.
\end{align*}
Für die Frequenzen von \num{100} bis \SI{0.02}{\kilo\hertz} wird ein Selektivverstärker verwendet. Bei diesem wird $V_\text{S} = 1$ und $Q = 10$ gewählt.

Mit dem Frequenzspektrum
\begin{align}
  W(\nu) = \frac{U_\text{a}^2}{R^2 \Delta \nu}
\end{align}
und nach rausrechnen aller Verstärkungen ergibt sich das Rauschspektrum aus Abbildung
\begin{figure}
  \centering
  \includegraphics[height=7cm]{build/plotRauschspektrumRein.pdf}
  \caption{Das Rauschspektrum der Hochvakuumdiode mit Reinmetallkathode.}
  \label{fig:plotRauschspektrumRein}
\end{figure}

\subsection{Untersuchung einer Röhre mit Oxydkathode}

Nun wird die Schaltung nach Abbildung \ref{fig:oxydschlatung}
aufgebaut.
Die Parameter lauten hier:
\begin{align*}
  V_= = 10 \quad V_\text{V} = 1000 \quad R = \SI{2200}{\ohm} \quad I_0 = \SI{0.9}{\ampere}.
\end{align*}
Die Messergebnisse sind in Tabelle \ref{tab:rauschspektrumoxid} zu finden. Dann wird das Rauschspektrum so wie im Kapitel zuvor bestimmt und in Abbildung \ref{fig:plotRauschspektrumOxid} gezeigt.
\begin{figure}
  \centering
  \includegraphics[height=7cm]{build/plotRauschspektrumOxid.pdf}
  \caption{Das Rauschspektrum der Röhre mit Oxydkathode.}
  \label{fig:plotRauschspektrumOxid}
\end{figure}
% Hier ist auch das Schrotrauschen in der theoretisch zu erwarteten Größe nach Gleichung \eqref{eqn:???} eingezeichnet.
Da das Schrotrauschen frequenzunabhängig ist, werden Werte die diesen Ursprung haben als konstant angenommen. So wird abgeschätzt, dass alle Werte der Frequenz $\nu > \SI{2}{\hertz}$
größtenteils alleinig durch das Schrotrauschen bestimmt werden und bei den restlichen diesem zusätzlich der Funkel-Effekt überlagert ist.

Ein nach Fehlerbalken gewichteter Fit nach der Funktion
\begin{align*}
  W_\text{F}(\nu) &= \text{C} \frac{1}{\nu^\alpha}
\intertext{ergibt folgende Ergebnisse:}
  C &= \SI{1.4(1)}{\ampere\squared\hertz}^{\alpha-1} \quad \num{0.84(7)}.
\end{align*}
Die Abweichung vom erwarteten Wert $\alpha = 1$ beträgt
\begin{align*}
  \Delta_\alpha = \SI{16(7)}{\percent}.
\end{align*}


Tabelle für copy and paste:
\begin{table}[h]
  \centering
  \begin{tabular}{S S}
    \toprule
    {$k$} & {$U\:/\:\si{\milli\volt}$}\\
    \midrule
    1 & 637.2\\
    3 & 212.4\\
    5 & 127.4\\
    7 & 91.03\\
    9 & 70.8\\
    \bottomrule
  \end{tabular}
  \caption{Amplituden Rechteckspannung.}
  \label{tab:rechtampl}
\end{table}
