\section{Auswertung}
\label{sec:Auswertung}

\subsection{Thermisches Rauschen eines Widerstands}

\paragraph{Einfache Schaltung}

Benutzt wird hier die Schaltung aus Abbildung \ref{fig:???}, die hier als einfache Schaltung bezeichnet wird. Um aus dem thermischen Rauschen eines Widerstands die Boltzmann Konstante $k$ zu bestimmen, muss zunächst eine Eichmesssung durchgeführt werden. Mit den Werten aus Tabelle \ref{tab:durchlasseinfach} ergibt sich normiert auf die Amplitude und die Nachverstärkung $V_\text{N}$ die Durchlasskurve aus Abbildung \ref{fig:plotdurchlasseinfach}.
\begin{figure}
  \centering
  \includegraphics{build/plotdurchlasseinfach.pdf}
  \caption{Durchlasskurve der Apparatur nach einfacher Schaltung aus Abbildung \ref{fig:???}.}
  \label{fig:plotdurchlasseinfach}
\end{figure}
Es werden folgende Einstellungsparameter gewählt:
\begin{align*}
  U_\text{Amplitude} = \SI{200}{\milli\volt} \quad \Delta \nu_\text{Bandfilter} = \num{1}-\SI{50}{\kilo\hertz}\quad V_= = 10 \quad V_\text{V} = 1000 \quad V_\text{N} = 1.
\end{align*}
Außerdem wird ein Abschwächer mit $V_\text{Abschwächung} = 1/1000$ vorgeschaltet. Die Vorverstärkung und die Gleichspannungsverstärkung werden nicht rausgerechnet, da sie in der folgenden Messung gleich bleiben. Das Integral der Kurve aus Abbildung \ref{fig:plotdurchlasseinfach} ergibt, berechnet mit der Simpsonregel, die Apparatekonstante:
\begin{align}
  A = \SI{495496239571}{\hertz}.
\end{align}

Damit kann nun die Boltzmann Konstante berechnet werden. Mit den Widerständen $R_1$ und $R_2$ wurden die Werte aus den Tabellen \ref{tab:widerstand1} und \ref{tab:widerstand2} aufgenommen. Die Vorverstärkung, Gleichspannungsverstärkung und der Filterbereich des Bandfilters werden gleichgelassen. Die Temperatur ist $T = \SI{23(2)}{\celsius}$.
Aus den Werten von $\overline{U_\text{a}^2}$ normiert auf die Nachverstärkung ergeben sich die Graphen in den Abbildungen \ref{fig:plotWiderstand1} und \ref{fig:plotWiderstand2}.
\begin{figure}
  \centering
  \includegraphics[height=7cm]{build/plotWiderstand1.pdf}
  \caption{Plot der Ausgangsspannung normiert auf die Nachverstärkung in Abhängigkeit vom Wert des Widerstands $R_1$.}
  \label{fig:plotWiderstand1}
\end{figure}
\begin{figure}
  \centering
  \includegraphics[height=7cm]{build/plotWiderstand2.pdf}
  \caption{Plot der Ausgangsspannung normiert auf die Nachverstärkung in Abhängigkeit vom Wert des Widerstands $R_2$.}
  \label{fig:plotWiderstand2}
\end{figure}
Ein linearer Fit nach
\begin{align}
  \overline{U_\text{a}^2} &= m R + b\\
\intertext{ergibt für $R_1$:}
  m_1 &= \SI{3.7(1)e-6}{\hertz\joule} & b_1 &= \SI{-5.4(8)e-4}{\volt\squared}
\intertext{und für $R_2$:}
  m_2 &= \SI{6.7(9)e-9}{\hertz\joule} & b_2 &= \SI{-4.4(4)e-9}{\volt\squared}
\end{align}
Aus Formel \eqref{eqn:???} ergibt sich für die Boltzmann Konstante:
\begin{align*}
\intertext{für $R_1$:}
  k &= \frac{m_1}{4 A T} = \SI{6.4(2)e-21}{\joule\per\kelvin}
\intertext{und für $R_2$:}
  k &= \frac{m_2}{4 A T} = \SI{1.14(2)e-23}{\joule\per\kelvin}.
\end{align*}

Zuletzt wird die Schaltung noch mit einem $\SI{0}{\ohm}$ Widerstand kurzgeschlossen. Dabei ergeben sich die Messwerte aus Tabelle \ref{tab:kurzschlusseinfach}.
\begin{table}
  \centering
  \begin{tabular}{S S}
    \toprule
    {$V_N$} & {$\overline{U_\text{a}^2}\:/\:\si{\milli\volt\squared}$}\\
    \midrule
    1 & -4\\
    2 & -4\\
    5 & -4\\
    10 & -4\\
    20 & -4\\
    50 & -4\\
    100 & -4\\
    200 & 9\\
    500 & 80\\
    1000 & 315\\
    \bottomrule
  \end{tabular}
  \caption{Amplituden der einfachen kurzgeschlossenen Schaltung.}
  \label{tab:kurzschlusseinfach}
\end{table}


\paragraph{Korrelatorschaltung}

Jetzt wird eine Schaltung nach Abbildung \ref{fig:???} aufgebaut; im Folgenden Korrelatorschaltung genannt. Zur Eichung wird erneut eine Durchlasskurve aufgenommen. Die eingestellten Parameter sind:
\begin{align*}
  U_\text{Amplitude} = \SI{500}{\milli\volt} \quad \nu_\text{m} = \SI{5}{\kilo\hertz}\quad V_= = 10 \quad V_\text{V} = 1000.
\end{align*}
Den beiden Eingängen der Vorverstärker wird jeweils ein Abschwächer mit $V_\text{Abschwächung} = 1/1000$ vorgeschaltet.
Die Messwerte für die Eichmesssung sind in Tabelle \ref{tab:durchlasskorr} eingetragen.
Diese Werte, normiert auf die Amplitude und die Nachverstärkung, sind als Durchlasskurve in Abbildung \ref{fig:plotdurchlasskorr} gezeigt.
\begin{figure}
  \centering
  \includegraphics{build/plotdurchlassKorrelator.pdf}
  \caption{Durchlasskurve der Apparatur nach der Korrelatorschaltung aus Abbildung \ref{fig:???}.}
  \label{fig:plotdurchlasskorr}
\end{figure}

Damit kann nun die Boltzmann Konstante berechnet werden. Mit den Widerständen $R_1$ und $R_2$ wurden die Werte aus den Tabellen \ref{tab:widerstand1korr} und \ref{tab:widerstand2korr} aufgenommen. Die Vorverstärkung, Gleichspannungsverstärkung und die Mittenfrequenz der Selektionsverstärker werden gleichgelassen. Die Temperatur ist $T = \SI{20(2)}{\celsius}$.
Aus den Werten von $\overline{U_\text{a}^2}$, normiert auf die Nachverstärkung sowie die hier eingestellte Verstärkungs des Selektionsverstärkers $V_\text{S} = 10$, ergeben sich die Graphen in den Abbildungen \ref{fig:plotWiderstand1korr} und \ref{fig:plotWiderstand2korr}.
\begin{figure}
  \centering
  \includegraphics[height=7cm]{build/plotWiderstand1Korr.pdf}
  \caption{Plot der Ausgangsspannung normiert auf die Nachverstärkung in Abhängigkeit vom Wert des Widerstands $R_1$.}
  \label{fig:plotWiderstand1korr}
\end{figure}
\begin{figure}
  \centering
  \includegraphics[height=7cm]{build/plotWiderstand2Korr.pdf}
  \caption{Plot der Ausgangsspannung normiert auf die Nachverstärkung in Abhängigkeit vom Wert des Widerstands $R_2$.}
  \label{fig:plotWiderstand2korr}
\end{figure}
Ein linearer Fit nach
\begin{align}
  \overline{U_\text{a}^2} &= m R + b\\
\intertext{ergibt für $R_1$:}
  m_1 &= \SI{1.166(7)e-10}{\hertz\joule} & b_1 &= \SI{-4.4(4)e-9}{\volt\squared}
\intertext{und für $R_2$:}
  m_2 &= \SI{1.10(1)e-9}{\hertz\joule} & b_2 &= \SI{2(1)e-8}{\volt\squared}
\end{align}
Aus Formel \eqref{eqn:???} ergibt sich für die Boltzmann Konstante:
\begin{align*}
\intertext{für $R_1$:}
  k &= \frac{m_1}{4 A T} = \SI{1.33(1)e-23}{\joule\per\kelvin}
\intertext{und für $R_2$:}
  k &= \frac{m_2}{4 A T} = \SI{1.26(2)e-23}{\joule\per\kelvin}.
\end{align*}

Zuletzt wird die Schaltung noch mit einem $\SI{0}{\ohm}$ Widerstand kurzgeschlossen. Dabei ergeben sich die Messwerte aus Tabelle \ref{tab:kurzschlusskorr}.
\begin{table}
  \centering
  \begin{tabular}{S S}
    \toprule
    {$V_N$} & {$\overline{U_\text{a}^2}\:/\:\si{\milli\volt\squared}$}\\
    \midrule
    50 & -5\\
    100 & -9\\
    200 & -26\\
    500 & -140\\
    1000 & -390\\
    \bottomrule
  \end{tabular}
  \caption{Amplituden der einfachen Korrelatorschaltung.}
  \label{tab:kurzschlusskorr}
\end{table}

\paragraph{Rauschzahl}

Jetzt kann für die Korrelatorschaltung noch die Rauschzahl bei $R = \SI{500}{\ohm}$ bestimmt werden. Da $\nu_\text{m} = \SI{5}{\kilo\hertz}$ gewählt wird, ist $\Delta \nu = \SI{700.7}{\hertz}$. Die Temperatur ist erneut $T = \SI{20(2)}{\celsius}$. Die Verstärkungsfaktoren sind:
\begin{align*}
  V_\text{N} = 200 \quad V_\text{S} = 10 \quad V_= = 10 \quad V_\text{V} = 1000.
\end{align*}
Damit ergibt sich mit dem Messwert aus Tabelle \ref{tab:widerstand1korr} und der Formel \eqref{eqn:???} die Rauschzahl
\begin{align}
  F = \num{0.98(5)}.
\end{align}

%\begin{figure}
%  \centering
%  \includegraphics{plotphase.pdf}
%  \caption{Plot.}
%  \label{fig:plot}
%\end{figure}

Tabelle für copy and paste:
\begin{table}[h]
  \centering
  \begin{tabular}{S S}
    \toprule
    {$k$} & {$U\:/\:\si{\milli\volt}$}\\
    \midrule
    1 & 637.2\\
    3 & 212.4\\
    5 & 127.4\\
    7 & 91.03\\
    9 & 70.8\\
    \bottomrule
  \end{tabular}
  \caption{Amplituden Rechteckspannung.}
  \label{tab:rechtampl}
\end{table}
