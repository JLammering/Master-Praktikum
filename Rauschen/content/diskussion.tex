\section{Diskussion}
\label{sec:Diskussion}

\subsection{Thermisches Rauschen eines Widerstands}

Die Durchlasskurve des Bandfilters hat die erwartete Gestalt, da sie in dem eingestellten Fenster von $\Delta \nu = 1-\SI{50}{\kilo\hertz}$ das meiste Signal durchlässt und Frequenzen außerhalb unterdrückt.
Die aus der Messung mit den Widerständen bestimmte Boltzmann-Konstante ergibt nur für $R_2$ einen annehmbaren Wert. Der Wert, der mit $R_1$ bestimmt wurde weicht um zwei Größenordnungen ab und legt so nahe, dass hier möglicherweise ungünstig gemessen wurde.
Das Kurzschließen der Schaltung zeigt, dass das Verstäkerrauschen erst bei hohen Verstärkungen eine Rolle spielt.

Auch die Durchlasskurve des Selektivverstärkers sieht wie erwartet aus. Er verstärkt die eingestellt Mittenfrequenz $\nu_\text{m} = \SI{5}{\kilo\hertz}$ um mehrere Größenordnungen stärker als Frequenzen, die größer oder kleiner sind.
Die bestimmten Boltzmann-Konstanten zeigen Abweichungen vom Theoriewert von weniger als $\SI{10}{\percent}$. Dies unterstreicht auch die Überlegenheit der Korrelatorschaltung gegenüber der einfachen SChaltung. Durch Kurzschließen der Schaltung zeigt sich erneut das Verhalten, dass Verstärkerrauschen erst bei großen Verstärkungen wichtig wird.
