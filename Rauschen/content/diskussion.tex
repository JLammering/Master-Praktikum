\section{Diskussion}
\label{sec:Diskussion}

\subsection{Thermisches Rauschen eines Widerstands}

Die Durchlasskurve des Bandfilters hat die erwartete Gestalt, da sie in dem eingestellten Fenster von $\Delta \nu = 1-\SI{50}{\kilo\hertz}$ das meiste Signal durchlässt und Frequenzen außerhalb unterdrückt.
Die aus der Messung mit den Widerständen bestimmte Boltzmann-Konstante ergibt nur für $R_2$ einen annehmbaren Wert. Der Wert, der mit $R_1$ bestimmt wurde weicht um zwei Größenordnungen ab und legt so nahe, dass hier möglicherweise ungünstig gemessen wurde.
Ein weiterer Hinweis, dass hier mit der Messung etwas nicht stimmte, ist auch dass die Rauschzahl um circa zwei Größenordnungen von der Erwartung abweicht.
Das Kurzschließen der Schaltung zeigt, dass das Verstäkerrauschen erst bei hohen Verstärkungen eine Rolle spielt.

Auch die Durchlasskurve des Selektivverstärkers sieht wie erwartet aus. Er verstärkt die eingestellt Mittenfrequenz $\nu_\text{m} = \SI{5}{\kilo\hertz}$ um mehrere Größenordnungen stärker als Frequenzen, die größer oder kleiner sind.
Die bestimmten Boltzmann-Konstanten zeigen Abweichungen vom Theoriewert von weniger als $\SI{10}{\percent}$. Dies unterstreicht auch die Überlegenheit der Korrelatorschaltung gegenüber der einfachen Schaltung. Durch Kurzschließen der Schaltung zeigt sich erneut das Verhalten, dass Verstärkerrauschen erst bei großen Verstärkungen wichtig wird.
Eine Fehlerquelle bei beiden Schaltungen ist die teils stark schwankende Anzeige der Rauschspannung.

Die Rauschzahl ist nah an der günstigsten Zahl von $F = 1$. Dies ist konsistent mit dem niedrigen Rauschen in Tabelle \ref{tab:kurzschlusskorr} bei $V_\text{N} = 200$.

\subsection{Untersuchung einer Hochvakuumdiode mit Reinmetallkathode}

Die Kennlinien für verschiedene Heizströme zeigen unterschiedliches Verhalten. Für $I_\text{heiz} = \SI{0.95}{\ampere}$ und $I_\text{heiz} = \SI{1}{\ampere}$ zeigt sich ein Ansteigen und das asymptotische Annähern an einen konstanten Plateauwert bei steigender Anodenspannung. Bei dem später verwendeten $I_\text{heiz} = \SI{0.9}{\ampere}$ ist der Anodenstrom relativ konstant über den gesamten Bereich der Anodenspannung. Deshalb ist davon auszugehen, dass man sich hier bereits anfangs im Sättigungsbereich befindet.

Die Elementarladung konnte nur mit großer Abweichung bestimmt werden. Die Messung hätte mit einem höheren Heizstrom vermutlich verbessert werden können, da dann insgesamt ein stärkeres Signal vorliegt und so kleine Abweichungen eine kleinere Rolle spielen.
% Eine Fehlerquelle ist, dass der Heizstrom wahrscheinlich nicht während der gesamten Messung konstant ist und so auch der Anodenstrom

\subsection{Untersuchung der Röhre mit Oxydkathode}

Die Abschätzung bei welchen Werten der Funkeleffekt eine Rolle spielt ist hier eine Fehlerquelle. Der Exponent des Funkeleffekts kann mit relativ kleiner Abweichung bestimmt werden.
