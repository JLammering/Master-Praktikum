 \section{Durchführung}
\label{sec:Durchführung}

\begin{enumerate}
  \item[a)] Im ersten Teil dieses Versuchs wird das thermische Widerstandsrauschen an
  einem einfachen Spektrometer untersucht, um die Rauschzahl des Spektrometers
  und die Boltzmannkonstante $k_\text{B}$ zu bestimmen. Dazu wird die Schaltung in
  Abbildung \ref{fig:rauschspektro} aufgebaut. Der Frequenzbereich am Bandfilter
  sollte dabei so gewählt sein, dass die Frequenzen $\SI{50}{\hertz}$ und $\SI{150}{\hertz}$
  nicht im Durchlassbereich liegen, da sonst große Störspannungen entstehen. Der obere
  Randwert für die Wahl des Frequenzbereichs sollte eine Frequenz von $\SI{500}{\kilo\hertz}$
  nicht überschreiten. Die Verstärkung des Vorverstärkers ist konstant bei $V_\text{V} = 1000$
  und die der anderen Verstärkerbauteile kann manuell eingestellt werden.

  Es wird zunächst mittels einer Eichmessung die Größe
  \begin{align}
    A = \int_{0}^{\nu_\text{max}} V_{=} V_\text{V}^2(\nu) V_\text{N}^2(\nu) \mathrm{d}\nu
  \end{align}
  bestimmt. Dazu wird an den Eingang des Spektrometers ein (rauschfreier) Abschwächer mit dem
  Verstärkungsfaktor $V_\text{A} = 1/1000$ und dahinter ein Sinusgenerator angeschlossen.
  Daraufhin wird die Ausgangsspannung am Voltmeter in Abhängigkeit von der Frequenz aufgenommen.

  Für die eigentliche Messung werden zwei zu untersuchende variable Widerstände mit unterschiedlichen
  Widerstandsbereichen an das Rauschspektrometer angeschlossen und die Ausgangsspannung $U_\text{A}^2$
  in Abhängigkeit von $R$ aufgenommen. Aus der Messreihe kann gemäß der vorgestellten theoretischen
  Grundlagen die Rauschzahl des Spektrometers und die Boltzmannkonstante $k_\text{B}$ ermittelt werden.

  \item[b)] Die im vorangegangenen Abschnitt vorgestellte Durchführung wird nun für das
  Rauschspektrometer nach dem Korrelatorprinzip wiederholt. Dabei wird anstatt des Bandfilters
  bei beiden Kanälen jeweils ein Selektivverstärker verwendet und die Eichmessung wiederholt.
  Dieser hat die Durchlasskurve
  \begin{align}
    U_\text{A}^2 = \frac1{Q^2} \cdot \frac1{\eta^2 + \frac1{\eta^2} + \frac1{Q^2} -2} \cdot U_\text{E}^2,
  \end{align}
  wobei $Q = 10$ die Güte am Selektivverstärker, $\eta = \nu/\nu_0$ und $\nu_0$ die Mittenfrequenz ist.

  \item[c)] Als nächstes wird das Schrotrauschen einer Hochvakuumdiode mit Reinmetallkathode untersucht,
  um die Elementarladung $\mathrm{e}_0$ zu bestimmen.
  Zunächst werden einige Kennlinien des Anodenstroms $I_0$ in Abhängigkeit der Anodenspannung $U_\text{An}$ bei
  konstantem Heizstrom aufgenommen, da die Diode für diese Messung im Sättigungsbereich arbeiten soll und dieser
  erst ermittelt werden muss. Zwar treten die Schwankungen des Schrotrauschens im Strom statt, jedoch besitzt
  die Diode einen Arbeitswiderstand, an dem letzendlich eine Spannung $U(t) = R I(t)$ abfällt. Da diese wesentlich
  größer als die zuvor untersuchten thermischen Rauschspannungen ist und damit das Verstärkerrauschen kaum eine Rolle spielt,
  genügt es, das einfache Spektrometer aus Abbildung \ref{fig:rauschspektro} anzuschließen um das verstärkte
  Spannungsquadrat in Abhängigkeit von der Mittenfrequenz aufzunehmen.

  \item[d)] Außerdem wird das Frequenzspektrum $W(\nu)$ des Schrotrauschens einer Hochvakuumdiode mit Reinmetallkathode
  aufgenommen. Dazu wird derselbe Versuchsaufbau wie in c) benutzt, mit der Änderung, dass das Bandfilter gegen
  einen Selektivverstärker ausgetauscht wird. Dabei ist zu beachten, dass die Mittenfrequenz $\nu_0$ und der
  Durchlassbereich $\Delta \nu$ korreliert sind.
  Bei der Messung der Ausgangsspannung in Abhängigkeit der Mittenfrequenz muss bei niedrigen Frequenzen
  $\nu_0 < \SI{500}{\hertz}$ die Zeitkonstante am RC-Tiefpass justiert werden. Außerdem sollte der Anodenstrom
  unter $\SI{4}{\milli\ampere}$ liegen um Störspannungen gering zu halten.

  \item[e)] Im letzten Teil dieses Experiments wird das Frequenzspektrum $W(\nu)$ einer Hochvakuumdiode mit Oxidmetallkathode
  aufgenommen. Dazu wird erneut das einfache Rauschspektrometer verwendet. Für hohe Frequenzen zwischen $\SI{100}{\kilo\hertz}$
  und $\SI{500}{\kilo\hertz}$ wird dabei das Bandfilter und für Frequenzen darunter bis $\SI{10}{\hertz}$ der Selektivverstärker
  verwendet. Die Diode mit Oxidmetallkathode wird nicht im Sättigungsbereich betrieben, sodass die Frequenzspektren der
  Aufgabenteile d) und e) nur qualitativ verglichen werden können.
\end{enumerate}
