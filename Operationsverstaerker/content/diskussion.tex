\section{Diskussion}
\label{sec:Diskussion}

Beim gegengekoppelten Verstärker folgen die Werte der erwarteten Theoriekurve. Nur bei der zweiten Widerstandskombination treten größere Abweichungen auf. Die Verstärkung ist unter Berücksichtigung der Fehler verträglich mit der Erwartung nach den Widerstandswerten. Die Konstanz des Verstärkungs-Bandbreite-Produkt kann anhand dieser Werte nicht erkannt werden. Auch der Wert für die Leerlaufverstärkung kann nur grob abgeschätzt werden, da die einzelnen bestimmten Werte sich stark unterscheiden. Die Phasen folgen ebenfalls der theoretischen Erwartung mit einem Plateau und dann einem linearen Abfall.

In den Oszilloskopbildern des Umkehr-Integrators ist die integrierende Wirkung gut zu erkennen. Die antiproportionale Abhängigkeit der Verstärkung von der Frequenz ist, anhand der Abbildung aber auch an dem mit $-1$ verträglichen Fitparameter $m$, gut zu erkennen. Die Abweichung von $k$ vom erwarteten Wert $RC$ ist verträglich bei Betrachtung der Fehler. Hier ist die geringe Statistik eine Fehlerquelle.

Der Umkehr-Differentiator zeigt in den Oszilloskopbildern das erwartete Verhalten. Der lineare Fit zeigt, dass die Messwerte dem theoretischen Zusammenhang folgen. Der bestimmte Parameter stimmt sehr genau mit dem erwarteten Wert überein.

Beim Schmitt-Trigger sind alle Abweichungen im verträglichen Bereich.

Die Abweichung der Frequenz beim Signalgenerator ist sehr hoch. \todo{Update nach richtiger Formel}.Die generierten Signale haben die erwünschte Form.

Die Abweichungen bei ent- und gedämpfter Schwingung zeigen die gute Übereinstimmung zwischen errechneten und aus dem Fit bestimmten Werten. Abweichungen können dadurch erklärt werden, dass $|\tau| \neq 1$. Im Oszilloskopbild kann gut die periodische Anregung und dann exponentielle Abnahme des Signals gesehen werden. Die exponentielle Zunahme beim entdämpften kann nicht gesehen werden, da die Spannung dort nach kurzer einen Sättigungswert erreicht.
