 \section{Durchführung}
\label{sec:Durchführung}

\subsection{Operationsverstärker}

Die im Versuch verwendeten Operationsverstärker mit dem Gehäuse DIL 8 haben den im Folgenden
beschriebenen Beschaltungsplan. Wenn der Operationsverstärker so gedreht wird, dass die Einkerbung
nach oben zeigt, so werden die acht Anschlüsse oben links beginnend bis unten links und anschließend
von unten rechts bis oben rechts durchnummeriert. Dabei ist Anschluss 2 der invertierende und
Anschluss 3 der nicht-invertierende Eingang. Die Betriebsspannungen werden an den Anschlüssen 4 ($-U_\text{B}$) und 7 ($+U_\text{B}$)
angelegt. Die Ausgangsspannung des Operationsverstärkers kann an Anschluss 6 abgegriffen werden. Die Anschlüsse
1, 5 und 8 sind in diesem Versuch unbelegt.

\subsection{Messprogramm}

\begin{enumerate}
  \item Im ersten Versuchsteil wird der Frequenzgang des in Abschnitt \ref{sec:linv} behandelten gegengekoppelten
  Linearverstärkers bei vier verschiedenen Widerstandspaaren, deren Quotient sich über über diverse
  Zehnerpotenzen erstrecken soll, ausgemessen. Dazu wird die Schaltung aus Abbildung \ref{fig:gegenkopplung} aufgebaut
  und an ihren Eingang ein Wechselspannungsgerät angeschlossen. Ferner wird die Ausgangsspannung des Linearverstärkers
  an ein Oszilloskop angeschlossen um zu kontrollieren, dass die oszillierende Eingangsspannung unverzerrt
  verstärkt wird. Es werden sowohl Verstärkungsfaktoren, als auch Phasenverschiebungen zwischen Eingangs-
  und Ausgangsspannungen aufgenommen.

  \item Als nächstes wird ein Umkehr-Integrator nach Abbildung \ref{fig:integrator} aufgebaut und jeweils ein
  Oszilloskopbild der Ausgangsspannung bei einer sinus-, rechteck- und dreieckförmigen Eingangsspannung aufgenommen.
  Anschließend wird der Verstärkungsfaktor in Abhängigkeit von der Frequenz bei der sinusförmigen Eingangsspannung
  ausgemessen um die Frequenzabhängigkeit nach Gleichung \eqref{eqn:int_aus} zu überprüfen.
  Hier ist es wichtig, die Offseteingangsspannung (DC-Offset) am Wechselspannungsgerät sorgfältig zu variieren, da
  auch kleine Gleichströme vom Integrator aufintegriert werden und entsprechend die Ausgangsspannung
  drastisch verändern.

  \item In diesem Versuchsteil werden dieselben Schritte wie im letzten durchgeführt, wobei der Umkehr-Integrator
  durch ein Umkehr-Differentiator ausgetauscht wird.
  Gleichströme am Differentiatoreingang sind hier weniger problematisch, da diese beim Differenzieren prinzipiell
  verschwinden.

  \item Im vierten Versuchsteil wird das Kippverhalten eines Schmitt-Triggers untersucht. Dazu wird
  zunächst die Schaltung nach Abbildung \ref{fig:schmitttrigger} aufgebaut, an den Eingang eine
  Sinusspannung angelegt und an den Ausgang ein Oszilloskop angeschlossen. Die Eingangsamplitude
  wird beginnend bei einer verschwindenden Spannung so lange erhöht, bis am Ausgang eine Rechteckspannung
  beobachtet wird. Anschließend wird für diesen kritischen Punkt die Eingangs- und Ausgangsspannungsamplitude
  aufgenommen.

  \item Der nächste Versuchsteil befasst sich mit der Generatorschaltung aus Abbildung \ref{fig:signalgen}.
  Zunächst wird der Schmitt-Trigger bzw. Rechteckgenerator aufgebaut, an den Eingang eine beliebige Wechselspannung mit ausreichend
  großer Amplitude angelegt und mit einem Oszillokop die Funktion überprüft. Ist diese Überprüfung erfolgreich, so
  wird der Umkehrintegrator dazugeschaltet und überprüft, ob am Ausgang eine Dreieckspannung anliegt. Anschließend
  wird das anliegende Wechselspannungsgerät ausgeschaltet und ein Bild der Rechteck- bzw. Dreieckspannung aufgenommen.

  \item Im letzten Versuchsteil wird der Signalgenerator nach Abbildung \ref{fig:expsin} untersucht. Für die
  Schaltung werden Kondensatoren mit $C \approx \SI{20}{\nano\farad}$ verwendet. Es wird ein Bild der
  entdämpften Schwingung und eine CSV-Datei der gedämpften Schwingung aufgenommen um diese anschließend
  auszuwerten. Bei der gedämpften Schwingung ist zu beachten, dass diese nicht von selbst anschwingt. Daher
  wird sie mit einer externen Rechteckspannung am invertierenden Eingang des zweiten Operationsverstärkers, wie auch
  in Abbildung \ref{fig:expsin} erkennbar, erzwungen.
\end{enumerate}
