\section{Theorie}
\label{sec:Theorie}

\subsection{Fehlerrechnung}

Für die Fehlerfortpflanzung bei Gleichungen mit $N$ fehlerbehafteten Größen
wird jeweils die Formel zur Gaußschen Fehlerfortpflanzung

\begin{equation*}
  \sigma = \sqrt{\sum_{i=1}^{N}\biggl(\frac{\partial f(x_{\g{i}})}{\partial x_{\g{i}}}
  \sigma_{\g{i}}\biggr)^2}
\end{equation*}
mit der jeweiligen Funktion $f(x_{\g{i}})$, den Messgrößen $x_{\g{i}}$ und den
zugehörigen Fehlern $\sigma_i$ verwendet.
Zur Berechnung des arithmetischen Mittels von $N$ Messwerten wird jeweils die
Formel

\begin{equation*}
  \bar{x} = \frac{1}{N}\sum_{i=1}^{N}x_{\g{i}}
\end{equation*}
mit den Messwerten $x_i$ benutzt.
Die Standardabweichung des Mittelwerts wird jeweils mit der Gleichung

\begin{equation*}
  \bar{\sigma} = \sqrt{\frac{1}{N-1}\sum_{i=1}^{N}(x_{\g{i}} - \bar{x})^2}
\end{equation*}
mit den $N$ Messwerten $x_i$ berechnet.


\cite{anleitung}
